\chapter{Introducción}
\label{cap:introducion}
\setcounter{page}{1}

% TODO: Descomentar cuando tengamos una frase que poner
\begin{flushright}
    \begin{minipage}[]{10cm}
        \emph{Basically, if reverse engineering is banned, then a lot of the open source community is doomed to fail}\\
    \end{minipage}\\

    Jon Lech Johansen, \textit{}\\
\end{flushright}

\vspace{1cm}

Antes de entrar en profundidad en el proyecto que en esta memoria se detalla me gustaria introducir formalmente el concepto de ``ingeniería inversa'' o conocida en el mundo de la
informatica como \textit{reverse engineering}. La ingineria inversa es el proceso de extraer conocimiento o el diseño por cualquier cosa que el humano haya hecho. Por lo tanto,
no es un concepto que solo acotemos dentro de la ingineria informatica si no que se puede aplicar a cualquier proceso ingineril. De hecho, este proceso es bastante similar al
metodo cientifico, con la unica diferencia que la ingineria inversa solo se aplica sobre cosas que ha hecho un humano y en el metodo cientifico lo aplicamos en fenomenos
naturales.

La ingineria inversa es usada normalmente para obtener conocimiendo desconocido o la filosofia del diseño cuando esta información no esta disponible, ya sea porque su propietario
ha decidido no compartirla o porque esta información esta perdida o destruida. \cite{alma991003132729706711}

La ingienería inversa aplicada al \textit{software} es el mismo concepto pero aplicada a programa en su formato binario, con el objetivo de obtener el codigo fuente en un lenguaje
de programación concreto y así poder obtener información como el diseño del programa. De hecho, la ingeniería inversa aplicada al \textit{software} requiere diferentes artes:
descifrado de codigos, resolución de puzzles, programación y analisis logico.

Las aplicaciones de la ingineria inversa son muy variadas, pero podemos destacar dos categorias: seguridad y desarrollo de \textit{software}.

\section{Contextualización}
\label{sec:contextualizacion}

% Corregido

Como bien he mencionado con anterioridad, este proyecto se enmarca dentro de un proyecto de final de Grado de Ingeniería Informática, en la especialidad de Tecnologías de la Información.
Este grado se imparte en la Facultad de Informática de Barcelona (FIB) y está dentro del contexto de la Universitat Politècnica de Catalunya (UPC).

En este trabajo se ha propuesto abordar una serie de competencias técnicas, que pasaré a detallar más adelante, que están estrechamente relacionadas con la especialidad cursada. Estas
son las competencias técnicas:

\begin{itemize}
    \item \textbf{CTI3.3} Diseñar, implantar y configurar redes y servicios.
    \item \textbf{CTI3.4} Diseñar \textit{software} de comunicaciones.
\end{itemize}

\section{Identificación del problema}
\label{sec:problema}

% Corregido

El proyecto busca atajar el problema que actualmente tenemos a la hora de aplicar ingeniería inversa sobre ejecutables. Actualmente, es una tarea compleja debido a que las optimizaciones
realizadas en tiempo de compilación hace que un código en C pierda totalmente la estructura en un código en ensamblador. Así mismo, la falta de una tabla de símbolos o información relevante
al nombramiento de funciones o variables hace que el código sea menos legible.

También sabemos, que todas estas modificaciones que se hace en un código compilado siguen ciertos patrones, ya que al final aplicamos métodos deterministas para hacer estas optimizaciones
u ocultaciones de código. Debido a estos patrones, creemos que con una cantidad de datos representativa y con la asistencia de inteligencia artificial podamos acelerar este proceso.

Por lo tanto, lo que se pretende en esta investigación es observar si con la ayuda de inteligencia artificial, concretamente a través de modelos del lenguaje, podemos obtener mejores
resultados que con las herramientas que hay actualmente en el mercado.

\section{Terminología y definiciones}
\label{sec:terminalogia}

% Corregido

En este apartado pasaremos a detallar ciertos términos reiterativos que veremos a lo largo del proyecto y que son de especial importancia. Las definiciones son las siguientes:

\begin{itemize}
    \item \textbf{Large Lenguage Model:} un LLM o \textit{Large Lenguage Model} son modelos \textit{Pre-trained Lenguage Models}\footnote{El concepto de pre entrenamiento en un modelo
                                        de lenguaje está relacionado con el aprendizaje por transferencia. La idea del aprendizaje por transferencia es reutilizar los conocimientos
                                        aprendidos en una o varias tareas y aplicarlos a tareas nuevas.} (PLM) a los cuales se han aumentado o el tamaño del modelo en sí o los datos.
                                        Con este aumento se dieron cuenta de que había una notoria mejora en términos de \textit{performance} y de la capacidad de los modelos en hacer
                                        ciertas tareas. \cite{ZhaoWayneXin2023ASoL}
    \item \textbf{Fine-tuning:} es una técnica de entrenamiento que consiste en reutilizar un modelo predefinido y preentrenado, de tal manera que ajustamos ciertas capas de la
                                red neuronal para obtener mejores resultados para nuestra tarea en específico.
\end{itemize}

\section{Actores implicados}
\label{sec:actores}

% Corregido

A continuación se pasarán a detallar a los actores implicados en el trabajo. Estos actores pueden participar de manera activa o pasiva dentro del proyecto, pero tiene una relevancia
importante dentro de este mismo:

\begin{itemize}
    \item \textbf{Investigador:} Este proyecto consta de un único investigador, Daniel García Estevez. Este se encargará de hacer tareas tales como la investigación de modelos a utilizar, 
                                 metodos de \textit{fine-tuning}, entre otras.
    \item \textbf{Director y codirector:} El director del trabajo de final de grado Alex Pajuelo Gonzalez perteneciente al CRAAX\footnote{Centre de Recerca d'Arquitectures Avançades de Xarxes}
                                        y profesor asociado a la UPC. Será el encargado de dirigir y supervisar este proyecto. Así mismo, el codirector del proyecto, Juan José Costa Prats
                                        perteneciente al CRAAX, ara tareas similares al director.
    \item \textbf{Experto en Inteligencia Artificial:} este proyecto constará con un experto en inteligencia artificial que dará soporte en temas más específicos que se salen fuera del alcance de
                                                    este proyecto. El experto es Jordi Nin Guerrero profesor titular en el Departamento de Operaciones, Innovación y Data Sciences en ESADE.
    \item \textbf{Colaboradores:} también contaremos con la ayuda de Xavi Verdú, profesor asociado de la UPC e investigador senior de PhD.
\end{itemize}