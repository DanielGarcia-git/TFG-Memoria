\chapter{Introducción}
\label{cap:introducion}

\setcounter{page}{1}

% Corregido 14/01/2024
% Revisado 14/01/2024
% TODO:daniel: Intentar añadi un poco mas de contexto

Es innegable que el \textit{software} ha cobrado una gran importancia en nuestra sociedad, tal es
la importancia que la industria del \textit{software} ha crecido al doble de la tasa de todas
las industrias en los últimos 10 años. Por dar un dato más concreto, en el año 2020
(el año de la pandemia) mientras la economía se contraía un 3.3\% la industria del
\textit{software} creció un 2.7\% y se espera que crezca más del doble del ritmo del PIB
mundial en los próximos cinco años. Teniendo que en septiembre de 2021 siete de las
diez empresas más valiosas del mundo son empresas de \textit{software}. \cite{IndustriaSoftware}

Esta importancia que ha cobrado el \textit{software} en nuestra sociedad ha hecho que
las características del \textit{software} sean cada vez más importantes, es decir, debemos
crear \textit{software} que sea seguro, fiable, escalable, etc. Para poder cumplir con
estas características se han creado una serie de metodologías y técnicas que nos ayudan
a poder cumplir con estas características. Una de estas técnicas es la ingeniería inversa.

La ingeniería inversa es un proceso en el cual a través de razonamiento deductivo se intenta
obtener información de un sistema, en este caso de un programa informático. Este razonamiento
deductivo requiere un nivel de conocimiento técnico alto por parte del que aplica la ingeniería
inversa, ya que se requiere de conocimientos de arquitectura de computadores, de sistemas
operativos, de lenguajes de programación, de compiladores, entre otros.

Es un proceso complejo que requiere de mucho tiempo, esfuerzo y conocimiento técnico alto.
Una de las técnicas que se están explorando para intentar automatizar el proceso, es la
aplicación de técnicas de aprendizaje automático.

Lo que intentaremos abordar en este proyecto es resolver la siguiente hipótesis: \textbf{¿utilizando
modelos de lenguaje del estilo GPT podemos automatizar la generación de código en C a partir de
un ejecutable?}

Todas estas cuestiones las trataremos dentro de esta memoria que se ha dividido en los
siguientes capítulos:

\begin{itemize}
    \item \textbf{Capítulo 1: Introducción} En este capítulo se introducirá el proyecto,
        se contextualizará y se detallará el contenido de esta memoria. Así mismo, se
        detallarán los actores implicados en el proyecto, los objetivos y subobjetivos
        que se quieren alcanzar y los posibles riesgos y obstáculos que se pueden encontrar.
    \item \textbf{Capítulo 2: Justificación} En este capítulo se detallarán el problema
        a solucionar, los estudios previos y la solución tomada.
    \item \textbf{Capítulo 3: Estado del arte} En este capítulo se detallará el estado
        del arte de la ingeniería inversa, las herramientas que existen actualmente y
        se detallarán los diferentes métodos de ingeniería inversa que existen. Así mismo,
        se hablará sobre el estado del arte de los modelos de lenguaje, métodos de aprendizaje
        y métodos de reducción de memoria a la hora de entrenarlos.
    \item \textbf{Capítulo 4: Fases de la Ingenieria inversa asistida por IA} En este
        capítulo se detallará la hoja de ruta que se ha seguido para el desarrollo de este
        proyecto.
    \item \textbf{Capítulo 5: Viabilidad de la Hipótesis a través de ChatGPT} En este capítulo
        se detallará la viabilidad de la hipótesis planteada a través de un modelo de lenguaje
        disponible e interactuable a través de internet.
    \item \textbf{Capítulo 6: Estrategia de entrenamiento} En este capítulo se detallará que
        modelo se ha elegido y la estrategia de entrenamiento que se ha utilizado para poder 
        entrenar el modelo.
    \item \textbf{Capítulo 7: Diseño e implementación del sistema de scripts} En este
        capítulo se detallará el diseño e implementación del sistema de scripts que se
        ha desarrollado para poder generar el \textit{dataset} que se utilizará para el
        entrenamiento del modelo.
    \item \textbf{Capítulo 8: Configuración y ejecución de los entornos de entrenamiento} En
        este capítulo se detallará la configuración y ejecución de los entornos de entrenamiento
        que se han utilizado para poder entrenar el modelo.
    \item \textbf{Capítulo 9: Resultados} En este capítulo se detallarán los resultados
        obtenidos durante el entrenamiento del modelo.
    \item \textbf{Capítulo 10: Planificación} En este capítulo se detallará la planificación
        del proyecto, así como la división en tareas y la estimación de los recursos
        necesarios. Así mismo, se detallará la metodología de trabajo que se ha utilizado.
    \item \textbf{Capítulo 11: Presupuesto} En este capítulo se detallará el presupuesto
        del proyecto, así como los recursos necesarios.
    \item \textbf{Capítulo 12: Sostenibilidad} En este capítulo se detallará el impacto
        que tendrá el proyecto en el medio ambiente, así como el impacto social y económico.
    \item \textbf{Capítulo 13: Conclusiones} En este capítulo se detallarán las conclusiones
        obtenidas durante el desarrollo del proyecto.
    \item \textbf{Capítulo 14: Trabajo futuro} En este capítulo se detallarán los posibles
        trabajos futuros que se pueden realizar a partir de este proyecto.
\end{itemize}

\section{Contextualización}
\label{sec:contextualizacion}

% Corregido 10/01/2024
% Revisado 14/01/2024

Este proyecto se enmarca dentro de un proyecto de final de Grado de Ingeniería
Informática, en la especialidad de Tecnologías de la Información. Este grado se
imparte en la Facultad de Informática de Barcelona (FIB) y está dentro del contexto
de la Universitat Politècnica de Catalunya (UPC).

En este trabajo se ha propuesto abordar una serie de competencias técnicas que 
stán estrechamente relacionadas con la especialidad cursada. Estas son las 
competencias técnicas:

\begin{itemize}
    \item \textbf{CTI3.1} Concebir sistemas, aplicaciones y servicios basados en
        tecnologías de red, incluyendo Internet, web, comercio electrónico,
        multimedia, servicios interactivos y computación ubicua. [En profundidad]
    \item \textbf{CTI3.3} Diseñar, implementar y configurar redes y servicios. [Bastante]
\end{itemize}

La competencia \textbf{CTI3.1} se desarrollará a través de la investigación de modelos de
lenguaje y de la implementación de un sistema de scripts que nos permitirá generar
el \textit{dataset} necesario para poder entrenar el modelo. La competencia \textbf{CTI3.3}
se desarrollará a través de la configuración de los entornos de entrenamiento y
la ejecución de los mismos.

\section{Actores implicados}
\label{sec:actores}

% Corregido 08/01/2024
% Revisado 14/01/2024

A continuación se pasarán a detallar a los actores implicados en el trabajo de final
de grado. Estos actores pueden participar de manera directa o indirecta dentro del
proyecto, pero tiene una relevancia importante dentro de este mismo:

\begin{itemize}
    \item \textbf{Investigador:} Este proyecto consta de un único investigador, \textbf{Daniel
        García Estevez}. Este se encargará de hacer tareas tales como la investigación
        de modelos a utilizar, metodos de \textit{fine-tuning}, entre otras. Así mismo,
        también se encargará del desarrollo de las herramientas necesarias para el proyecto.
    \item \textbf{Director y codirector:} El director del trabajo de final de grado,\textbf{ Alex
        Pajuelo Gonzalez}. Será el encargado de dirigir y supervisar este proyecto. Así
        mismo, el codirector del proyecto, \textbf{Juan José Costa Prats} ara tareas similares al director.
    \item \textbf{Experto en Inteligencia Artificial:} este proyecto constará con un experto en
        inteligencia artificial que dará soporte en temas más específicos que se salen fuera del
        alcance de este proyecto. El experto es \textbf{Jordi Nin Guerrero} profesor titular en el
        Departamento de Operaciones, Innovación y Data Sciences en ESADE.
    \item \textbf{Colaboradores:} también contaremos con la ayuda de \textbf{Xavi Verdú}.
\end{itemize}

\section{Objetivos y subobjetivos}
\label{sec:objetivos}

% Corregido 14/01/2024
% Revisado 14/01/2024

Como he introducido en el capítulo \ref{cap:introducion} nuestro principal
objetivo en este proyecto es poder generar código en C compilable, realista y fiel
al original, partiendo de un fichero ejecutable. Para poder alcanzar este objetivo
nos asistiremos con inteligencia artificial de tal manera que podamos conseguir
resultados óptimos y rápidos.

Para poder alcanzar este objetivo se ha decidido dividirlo en subobjetivos, de tal
manera que podamos ir dando pasos hasta el objetivo principal. Los subobjetivos definidos
son los siguientes:

\begin{enumerate}
    \item Despliegue y entrenamiento de un modelo de redes neuronales ya existente para
        poder asistirnos a generar código en C a partir de un ensamblado
    \item Estudio y pruebas con modelos online tipo \textit{chatbot} para obtener de forma rápida
        unos resultados preliminares.
    \item Creación de un \textit{dataset} para poder entrenar el modelo
    \item Obtener conclusiones de la viabilidad de la hipótesis planteada
\end{enumerate}

En el primer subobjetivo nos proponemos poder desplegar en un entorno con las capacidades
técnicas suficientes y disponibles para poder ejecutar un modelo de red neuronal que nos pueda asistir en
las tareas de aplicar ingeniería inversa sobre un fichero ejecutable.

En el segundo subobjetivo nos proponemos obtener resultados preliminares que nos faciliten
unos resultados orientativos para poder estimar la calidad de los resultados que podamos
obtener en nuestro objetivo principal. Para ello, nos asistiremos de modelos tipos \textit{chatbot}
que podemos interactuar desde internet y así poder obtener resultados inmediatos, sin necesidad
de desplegar un entorno dedicado.

En el tercer subobjetivo nos proponemos automatizar la creación de un \textit{dataset} que
nos permita entrenar el modelo de redes neuronales. Para ello, crearemos un sistema de scripts
que se encargará de la creación de este \textit{dataset}.

En el cuarto y último subobjetivo nos proponemos obtener conclusiones de la viabilidad de la
hipótesis planteada. Para ello, nos basaremos en los resultados obtenidos en los subobjetivos
anteriores.

\section{Posibles riesgos y obstáculos}
\label{sec:riesgos}

% Corregido 14/01/2024
% Revisado 14/01/2024

Los obstáculos que se presentan en este proyecto son elevados debido a la incertidumbre
que existe acerca de la hipótesis planteada. Dado que los antecedentes son escasos, y
la tecnología usada encuentra en auge y aun en desarrollo, los posibles obstáculos
o resultados no esperados que podamos hallar durante la investigación y el desarrollo
son altamente significativos.

Se han establecido los siguientes riesgos:

\begin{itemize}
    \item \textbf{Resultados no deseados:} podemos encontrarnos que a la hora de querer
        generar código C a partir de un ejecutable el modelo no sea capaz de procesar estos datos
        y, por lo tanto, nos dé una salida que no es la esperada o deseada.
    \item \textbf{Insuficiencia de recursos:} los modelos que utilizaremos son modelos
        muy grandes que necesita mucha memoria para poderse ejecutar, aunque sabemos que
        existen modelos más pequeños y que necesitan menos recursos, podemos encontrarnos de
        que no sea suficiente esta reducción.
\end{itemize}

\subsection{Plan de contingencia}
\label{subsec:planContingencia}

% Corregido 05/01/2024

Tal y como se ha descrito en la sección \ref{sec:riesgos} los riesgos que se asumen en este
proyecto son altos, por lo tanto, es necesario tener un plan de contingencia que describe
el procedimiento a aplicar cuando el riesgo se produzca. A continuación se detallarán
el plan de contingencia elaborado para este proyecto.

Como se puede observar en la tabla \ref{tab:plan_contignecia} se detalla por cada riesgo
la prevención y la respuesta que se dará a este riesgo. La prevención es la acción que se
llevará a cabo para intentar evitar que el riesgo se produzca. La respuesta es la acción
que se llevará a cabo en caso de que el riesgo se produzca.

\begin{table}[H]
    \centering
    \resizebox{\textwidth}{!}{%
    \begin{tabular}{|l|l|l|}
    \hline
    \rowcolor[HTML]{8EA9D8} 
    Riesgo                    & Prevención                                                                                                                                                                                                & Respuesta                                                                                                                                                                                                                                       \\ \hline
    Resultados no deseados    & \begin{tabular}[c]{@{}l@{}}\textbf{Validación rigurosa:} Validar el diseño del dataset\\ y de los modelos utilizados\\ \\ \textbf{Desarrollo iterativo:} Utilizar metodologías agiles\\ de desarrollo\end{tabular} & \begin{tabular}[c]{@{}l@{}}\textbf{Revisión y ajuste:} Ajustar el sistema y el entrenamiento \\ cuando no se ajusten los resultados a lo esperado\\ \\ \textbf{Colaboración con expertos:} colaboración con los expertos\\ asociados a este proyecto\end{tabular} \\ \hline
    Insuficiencia de recursos & \begin{tabular}[c]{@{}l@{}}\textbf{Evaluación de recursos:} evaluar lo que se necesitan\\ de recursos\\ \\ \textbf{Planificación de recursos:} asegurar la disponibilidad\\ de los recursos\end{tabular}           & \begin{tabular}[c]{@{}l@{}}\textbf{Escalabilidad de recursos:} en caso de que sean insuficientes,\\ mirar métodos para poder escalarlos\\ \\ \textbf{Optimización de modelos:} optimización de los modelos\\ para que estos consuman menos recursos\end{tabular}  \\ \hline
    \end{tabular}%
    }
    \caption[Plan de contingencia donde se detalla por cada riesgo la prevención y respuesta que se da a este]{Plan de contingencia donde se detalla por cada riesgo la prevención y respuesta que se da a este (Elaboración propia)}
    \label{tab:plan_contignecia}
\end{table}

