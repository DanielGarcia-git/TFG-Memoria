\chapter{Configuración y ejecución de los entornos de entrenamiento}
\label{cap:configuracion_ejecucion}

% Corregido 08/01/2024
% TODO:daniel: Revisar la configuración y ejecución de los entornos de entrenamiento

En este capítulo se detalla la configuración que se ha realizado para la ejecución de los
del modelo de lenguaje y qué plataformas se han utilizado para la ejecución de los mismos.
En la realización de este proyecto se ha utilizado dos entornos de ejecución diferentes:

\begin{itemize}
    \item \textbf{Entorno local:} se ha utilizado el cluster \textit{sert} del departamento
        de Arquitectura de Computadores de la Universitat Politècnica de Catalunya.
    \item \textbf{Entorno en la nube:} se ha utilizado una plataforma de entrenamiento en la nube
        llamada Lightning IA Studio.
\end{itemize}

\section{Configuración}

\subsection{Cluster AC: \textit{Sert}}
\label{subsec:cluster_ac}

El cluster de AC es un cluster de computación de alto rendimiento que dispone el departamento
de Arquitectura de Computadores de la Universitat Politècnica de Catalunya. Este cluster dispone
de un nodo con GPU's que recibe el nombre de \textit{sert-2001}. Este nodo tiene las siguientes
características técnicas:

\begin{itemize}
    \item 1 nodo 2x Intel Xeon Silver 4210R a 2.40Ghz
    \item 128 GB de memoria RAM
    \item 2 discos duros de 480GB SSD
    \item 2 discos duros de 2TB NVME
    \item 2 tarjetas de red 10 Gigabit Ethernet
    \item 8 GPU NVIDIA RTX 2080TI 11 GB GDDR6 PCIe
\end{itemize}

La conexión con el cluster se realiza a través de una VPN proporcionada por el departamento
de Arquitectura de Computadores. Una vez conectado a la VPN, se puede acceder al cluster
a través de SSH. El cluster dispone de un sistema de colas de ejecución de trabajos. Para
enviar un trabajo a ejecutar en el cluster se utiliza el comando \textit{sbatch}. En la figura
\ref{tab:colaGPU} tenemos la definición de las colas que disponemos para este nodo. Así mismo,
nos podemos conectar de forma interactiva a un nodo del cluster utilizando el comando
\textit{srun}.

\begin{table}[H]
    \centering
    \resizebox{\textwidth}{!}{%
    \begin{tabular}{|l|l|l|l|l|}
    \hline
    \rowcolor[HTML]{8EA9D8} 
    Cola        & Disponibilidad de GPU's & Tiempo maximo de ejecución & Otros limites                                                                                    & Observaciones    \\ \hline
    big\_gpu    & máximo 8, mínimo 5      & 4 horas                    &                                                                                                  &                  \\ \hline
    medium\_gpu & máximo 4, mínimo 2      & 2 dias                     & \begin{tabular}[c]{@{}l@{}}1 trabajo en ejecución\\ mínimo entre todos los usuarios\end{tabular} &                  \\ \hline
    small\_gpu  & máximo 1, mínimo 1      & 3 dias                     & \begin{tabular}[c]{@{}l@{}}3 trabajos en ejecución\\ máximo por usuario\end{tabular}             & cola por defecto \\ \hline
    \end{tabular}%
    }
    \caption[Colas definidas en el nodo de GPU's]{Colas definidas en el nodo de GPU's (Elaboración propia)}
    \label{tab:colaGPU}
\end{table}

Dentro del cluster disponemos de un directorio personal donde podemos almacenar nuestros
archivos. Este directorio se encuentra en la ruta \textit{/scratch/nas/3/danielg/}.

\subsubsection{Configuración del entorno}
\label{subsubsec:configuracion_entorno}

Como se ha mencionado en el capítulo \ref{cap:estrategia_entrenamiento}, se ha utilizado
la plataforma de desarrollo \textit{lit-gpt}. Para la configuración del entorno se ha
a seteado un entorno virtual de Python y instalado todas las dependencias que requiere
la plataforma. Para la instalación de las dependencias se ha utilizado el gestor de paquetes
de Python \textit{pip}.

\subsubsection{Limitaciones}
\label{subsubsec:limitaciones}

% Corregido 08/01/2024
% TODO:daniel: Revisar limitaciones del cluster AC

Las limitaciones que tiene este entorno de ejecución son las siguientes:

\begin{itemize}
    \item \textbf{Configuración de las GPU's:} la memoria que disponen las GPU's es de 11GB.
        Esto hace que no se puedan entrenar modelos de lenguaje muy grandes. Por lo tanto,
        se deberán de aplicar técnicas para reducir el tamaño de los modelos de lenguaje a la
        hora de entrenarlos.
    \item \textbf{Configuración del entorno:} es un entorno que no se ha configurado previamente
        para el entrenamiento de modelos de lenguaje. Por lo tanto, se deberán de configurar, instalar
        y comprobar el correcto funcionamiento de las librerías necesarias para el entrenamiento.
\end{itemize}

\subsection{Lightning IA Studio}
\label{subsec:lightning_ia_studio}

% Corregido 08/01/2024
% TODO:daniel: Revisar Lightning IA Studio

Lightning IA Studio es una plataforma de entrenamiento en la nube que permite entrenar
modelos de lenguaje de manera sencilla. Lo que promete es un entorno de entrenamiento
sencillo, rápido, escalable y sin necesidad de configurar el entorno. Esta plataforma
se encuentra en fase beta y puedes acceder al instante si se tienen cuenta acabada en
edu. En la figura \ref{fig:lightning_ia_studio} tenemos una captura de pantalla de la
plataforma.

\begin{figure}[H]
    \begin{center}
      \includegraphics[scale=0.3]{figuras/Capitulo_08/Lightning_Studio.png}
    \end{center}
    \caption[Captura de pantalla de la interfaz de usuario de Lightning Studio]{Captura de pantalla de la interfaz de usuario de Lightning Studio (Elaboración propia)}
    \label{fig:lightning_ia_studio}
\end{figure}

Así mismo, nos ofrece un entorno amigable y sencillo, donde por ejemplo podemos cambiar de
un entorno de CPU a un entorno de GPU y viceversa con un sencillo clic. Nos ofrece estos dos
entornos con las siguientes características:

\begin{itemize}
    \item \textbf{Entorno CPU}
    \begin{itemize}
        \item Un entorno con 4 núcleos de CPU pensado para tareas menos pesadas.
        \item Un entorno con 32 núcleos de CPU pensado para tareas más pesadas como la preparación de conjuntos de entrenamientos muy grandes.
    \end{itemize}
    \item \textbf{Entorno GPU: }
    \begin{itemize}
        \item 1 o 4 NVIDIA A10 Tensor core optimizada para trabajos de IA
        \item 1 o 4 NVIDIA V100 Tensor core optimizada para trabajos de IA
        \item 1 o 4 NVIDIA T4 Tensor core optimizada para trabajos de IA
    \end{itemize}
\end{itemize}

El servicio de Lightning IA Studio sigue un modelo de pago por uso. Donde se utilizan
créditos que equivalen a horas de GPU. Concretamente, 15 créditos nos da para 6 horas de
uso en la GPU. Así mismo, nos ofrecen 15 créditos gratuitos cada mes. \cite{lightningiaPricing}

\subsubsection{Limitaciones}
\label{subsubsec:limitaciones}

% Corregido 08/01/2024
% TODO:daniel: Revisar limitaciones de Lightning IA Studio

La principal limitación que tiene esta plataforma es que es un servicio de pago que dentro
del contexto de este proyecto no se ha asumido debido a que se ha utilizado como refuerzo, es
decir, donde no se ha llegado con el clúster del departamento de Arquitectura de Computadores
se ha llegado con esta solución.

\section{Ejecución}
\label{sec:ejecucion}

En ambos entornos se ha ejecutado el mismo conjunto de entrenamiento. Este conjunto de
entrenamiento se ha generado con el sistema de scripts que se ha desarrollado en el capítulo
\ref{cap:diseñoImplentacion_scripts}.

\subsection{Cluster AC: \textit{Sert}}
\label{subsec:cluster_ac_ejecucion}



\subsection{Lightning IA Studio}
\label{subsec:lightning_ia_studio_ejecucion}


