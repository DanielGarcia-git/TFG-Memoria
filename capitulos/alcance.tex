\chapter{Alcance}
\label{cap:alcance}

% Corregido

Al ser un proyecto que puede ser complejo y largo, definiremos el alcance y los objetivos que se deberán de cumplir para considerar que las hipótesis que nos
hemos planteado estén satisfechas.

\section{Objetivos}
\label{sec:objetivos}

% Corregido

Este proyecto tiene como objetivo principal investigar el uso de inteligencias artificiales para aplicar ingeniería inversa a programas.
En otras palabras, nos proponemos utilizar inteligencias artificiales tipo ChatGPT o similares para generar un código en C a partir única
y exclusivamente del ejecutable o de un archivo objeto.

Para poder alcanzar este objetivo se ha decidido dividirlo en subobjetivos, de tal manera que podamos ir dando pasos hasta el objetivo principal.
Los subobjetivos definidos son los siguientes:

\begin{itemize}
    \item Pruebas de uso con ChatGPT Plus, utilizando modelos como el GPT-3.5 y GPT-4
    \item Pruebas de uso con un modelo más pequeño denominados LoRA\footnote{LoRA (\textit{Low-Rank Adaptation}) es una técnica de \textit{fine-tuning} para modelos
        preentrenados en la cual propone utilizar una matriz de rangos baja de tal manera que reducimos los tiempos y memoria de GPU necesarios para el entrenamiento.}
        y haciendo \textit{fine-tuning} para nuestra tarea en específico.
\end{itemize}

El objetivo de realizar pruebas con ChatGPT es obtener una aproximación de los resultados que podamos obtener con nuestro modelo. ChatGPT nos
ofrece un entorno rápido y sencillo para poder hacer pruebas y analizar los resultados que podamos obtener.

Una vez realizadas las pruebas con ChatGPT, independientemente del resultado obtenido, se procederá a desplegar una inteligencia artificial más pequeña,
denominadas LoRa, en la cual se procederá a hacer diferentes métodos de \textit{fine-tuning}de tal manera que podemos aumentar la calidad de los resultados
obtenidos. La principal razón por la cual utilizamos modelos LoRa se debe a una cuestión técnica. Estos modelos están optimizados para obtener resultados
similares a un modelo normal como GPT-3.5 pero utilizando una cantidad menor de recursos hardware.

\section{Posibles riesgos y obstaculos}
\label{sec:riesgos}

% Corregido

Los riesgos que se asumen en este proyecto son altos debido a la incertidumbre que hay sobre este tema. Dado que los antecedentes son prácticamente nulos y dada una tecnología
que está en auge y aun en desarrollo, los posibles problemas o resultados no esperados que nos podamos encontrar durante la investigación y el desarrollo son altos.

Alguno de los posibles obstáculos que nos podemos encontrar son:

\begin{itemize}
    \item \textbf{Resultados no deseados:} podemos encontrarnos que a loa hora de querer generar código C a partir de un ejecutable el modelo no sea capaz de procesar estos datos
                                        y, por lo tanto, nos dé una salida que no es la esperada.
    \item \textbf{Insuficiencia de recursos:} los modelos que utilizaremos son modelos muy grandes que necesita mucha memoria de GPU para poderse ejecutar, aunque sabemos que
                                        existen modelos más pequeños y que necesitan menos recursos, podemos encontrarnos de que no sea suficiente esta reducción.
\end{itemize}