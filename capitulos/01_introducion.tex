\chapter{Introducción}
\label{cap:introducion}

\setcounter{page}{1}

\begin{flushright}
    \begin{minipage}[]{10cm}
        \emph{Sometimes, the best way to advance is in reverse}\\
    \end{minipage}\\

    Eldad Eilam \textit{}\\
\end{flushright}

\vspace{1cm}

% Corregido

Antes de entrar en profundidad en el proyecto que en esta memoria se detalla me gustaría introducir formalmente el concepto de ''ingeniería inversa'' o conocida en el mundo de la
informática como \textit{reverse engineering}. La ingeniería inversa es el proceso de extraer conocimiento o el diseño por cualquier cosa que el humano haya hecho. Por lo tanto,
no es un concepto que solo acotemos dentro de la ingeniería informática, sino que se puede aplicar a cualquier proceso ingenieril. De hecho, este proceso es bastante similar al
método científico, con la única diferencia que la ingeniería inversa solo se aplica sobre cosas que ha hecho un humano y en el método científico lo aplicamos en fenómenos
naturales.

La ingeniería inversa es usada normalmente para obtener conocimiento desconocido o la filosofía del diseño cuando esta información no está disponible, ya sea porque su propietario
ha decidido no compartirla o porque esta información está perdida o destruida. \cite{alma991003132729706711}

La ingeniería inversa aplicada al \textit{software} es el mismo concepto, pero aplicada a programa en su formato binario, con el objetivo de obtener el código fuente en un lenguaje
de programación concreto y así poder obtener información como el diseño del programa. De hecho, la ingeniería inversa aplicada al \textit{software} requiere diferentes artes:
descifrado de códigos, resolución de puzzles, programación y análisis lógico.

Las aplicaciones de la ingeniería inversa son muy variadas, pero podemos destacar dos categorías: seguridad y desarrollo de \textit{software}.

Sus aplicaciones en seguridad son muy variadas, pero normalmente se le relaciona con \textit{malware}, algoritmos criptográficos y auditorias sobre binarios.

Dentro del mundo del \textit{software} malicioso vemos que la ingeniería inversa se utiliza en dos aspectos diferentes. Desde el punto de perspectiva de los que desarrollan
\textit{software} malicioso utilizan la ingeniería inversa para poder encontrar vulnerabilidades en los programas que quieren infectar. En cambio, desde el punto de vista de los
desarrolladores de antivirus, utilizan la ingeniería inversa para poder diseccionar y analizar cada programa malicioso.

También se puede aplicar ingeniería inversa sobre algoritmos criptográficos, de tal manera que podamos averiguar que tan seguro es ese mensaje encriptado, o incluso en caso de utilizar
algoritmos basados en llaves, en los cuales la especificación del algoritmo de encriptación es conocido, pero que cada implementación específica puede variar, encontrar vulnerabilidades
en estos algoritmos.

Y por último, también la ingeniería inversa se aplica para realizar auditorias sobre binarios, de tal manera que podamos detectar si un programa es seguro o no, encontrar sus vulnerabilidades
para poder corregirlas. Por lo tanto, cuanto mejores sean las herramientas de ingeniería inversa que se apliquen, podremos encontrar con mucha más eficacia problemas de seguridad en
\textit{software} propietario.

Como he mencionado con anterioridad, la ingeniería inversa también se aplica dentro del desarrollo de \textit{software}, estas aplicaciones las podemos encontrar en diferentes etapas
por ejemplo cuando disponemos de un \textit{software} propietario y a documentación es escasa, el uso de herramientas de ingeniería inversa nos podrían ayudar a conseguir más interoperabilidad
con el \textit{software} propietario.

Otras de las aplicaciones, y que he mencionado anteriormente, es el comprobar la robustez y calidad de un programa informático, de tal manera que se pueda corregir posibles problemas
de seguridad o problemas funcionales.

Una vez introducido el concepto de ingeniería inversa de manera general, creo que es interesante contestar a la siguiente pregunta: \textbf{¿Es la ingeniería inversa legal?}

Como hemos podido ver en los párrafos anteriores, muchas veces aplicamos ingeniería inversa sobre un \textit{software} propietario, algunos con fines de conocer vulnerabilidades y corregirlas
y otros con fines de encontrar estas vulnerabilidades para tener un punto de ataque. En otras palabras, la legalidad de la aplicación de ingeniería inversa dependerá mucho de para
que lo estamos aplicando, la finalidad, y sobre que lo estamos aplicando.

En conclusión, la ingeniería inversa es un proceso complejo y que requiere de muchas habilidades y que sus aplicaciones son muy variadas, desde auditorias sobre binarios hasta mejorar
la interoperabilidad de dos programas.

\section{Contextualización}
\label{sec:contextualizacion}

% Corregido

Como bien he mencionado con anterioridad, este proyecto se enmarca dentro de un proyecto de final de Grado de Ingeniería Informática, en la especialidad de Tecnologías de la Información.
Este grado se imparte en la Facultad de Informática de Barcelona (FIB) y está dentro del contexto de la Universitat Politècnica de Catalunya (UPC).

En este trabajo se ha propuesto abordar una serie de competencias técnicas, que pasaré a detallar más adelante, que están estrechamente relacionadas con la especialidad cursada. Estas
son las competencias técnicas:

\begin{itemize}
    \item \textbf{CTI3.3} Diseñar, implantar y configurar redes y servicios.
    \item \textbf{CTI3.4} Diseñar \textit{software} de comunicaciones.
\end{itemize}

\section{Identificación del problema}
\label{sec:problema}

% Corregido

Como he explicado en el capítulo \ref{cap:introducion} la ingeniería inversa es aplicada en una infinidad de campos y de los cuales hoy en día es utilizada. Pero como he mencionado
aplicar ingeniería inversa no es una tarea fácil, ya que actualmente los programas que podemos encontrar en el mercado son de tal complejidad que aplicar ingeniería inversa sobre la
totalidad del programa supone horas y horas de trabajo, teniendo el riesgo de que los resultados obtenidos no se asemejen a la realidad.

Esta complejidad no solo viene dada por el gran tamaño de los programas actuales, sino de las técnicas que se utilizan para poder ocultar aún más el programa original. Algunas
de estas técnicas son \textit{constant blinding}\footnote{Esta técnica consiste en poner un valor aleatorio a las constantes (a través de operaciones como la XOR)}, cambiar el
encoding de las variables\footnote{Busca la ocultación del valor de la variable cambiando su representación de datos}, agregación de datos\footnote{Busca agrupar variables del mismo
tipo, por ejemplo bajo un \textit{struct}}, separación de datos\footnote{Al contrario que la agregación de datos, esta busca separar los datos en unidades más pequeñas, por ejemplo
de un \textit{short} a un \textit{char}}, \textit{dead code insertion}\footnote{Agregar código redundante al programa}, \textit{loop unrolling}\footnote{Es una tecnica que aplican
los compiladores que además de disminuir el coste computacional del programa hace que el código sea menos legible}, entre otras. \cite{TecnicasIlegibleBinario}

También a la hora de aplicar ingeniería inversa debemos sumar la complejidad de las aplicaciones distribuidas, es decir, antes, cuando disponíamos de un programa disponíamos de su
código en su totalidad, aunque este fuese en forma de un binario. Con las aplicaciones distribuidas, el programa se ha segmentado en diferentes partes y estas partes cada una se
puede encontrar en una máquina diferente, provocando que no dispongamos a veces del todo el programa y, por lo tanto, de todo su diseño y lógica.

En consecuencia, debido a las técnicas de ocultación de código, las optimizaciones que el compilador puede aplicar sobre el código, el gran tamaño de los programas modernos y el auge
de las aplicaciones distribuidas hacen que la tarea de aplicar ingeniería inversa sobre un programa sea muy compleja y que las soluciones actuales, que detallaré en la sección
\ref{sec:alternativas}, no son capaces de dar buenos resultados, sino que dan una especie de pseudocódigo que nos puede ayudar a entender la lógica del programa.

Todos estos factores contribuyen a que aplicar ingeniería inversa sea muy complejo y costoso y, por lo tanto, inviable en muchos casos para ciertos escenarios.

\section{Terminología y definiciones}
\label{sec:terminalogia}

% Corregido

En este apartado pasaremos a detallar ciertos términos reiterativos que veremos a lo largo del proyecto y que son de especial importancia. Las definiciones son las siguientes:

\begin{itemize}
    \item \textbf{Large Lenguage Model:} un LLM o \textit{Large Lenguage Model} son modelos \textit{Pre-trained Lenguage Models}\footnote{El concepto de pre entrenamiento en un modelo
                                        de lenguaje está relacionado con el aprendizaje por transferencia. La idea del aprendizaje por transferencia es reutilizar los conocimientos
                                        aprendidos en una o varias tareas y aplicarlos a tareas nuevas.} (PLM) a los cuales se han aumentado o el tamaño del modelo en sí o los datos.
                                        Con este aumento se dieron cuenta de que había una notoria mejora en términos de \textit{performance} y de la capacidad de los modelos en hacer
                                        ciertas tareas. \cite{ZhaoWayneXin2023ASoL}
    \item \textbf{Fine-tuning:} es una técnica de entrenamiento que consiste en reutilizar un modelo predefinido y preentrenado, de tal manera que ajustamos ciertas capas de la
                                red neuronal para obtener mejores resultados para nuestra tarea en específico.
\end{itemize}

\section{Actores implicados}
\label{sec:actores}

% Corregido

A continuación se pasarán a detallar a los actores implicados en el trabajo. Estos actores pueden participar de manera activa o pasiva dentro del proyecto, pero tiene una relevancia
importante dentro de este mismo:

\begin{itemize}
    \item \textbf{Investigador:} Este proyecto consta de un único investigador, Daniel García Estevez. Este se encargará de hacer tareas tales como la investigación de modelos a utilizar, 
                                 metodos de \textit{fine-tuning}, entre otras.
    \item \textbf{Director y codirector:} El director del trabajo de final de grado Alex Pajuelo Gonzalez perteneciente al CRAAX\footnote{Centre de Recerca d'Arquitectures Avançades de Xarxes}
                                        y profesor asociado a la UPC. Será el encargado de dirigir y supervisar este proyecto. Así mismo, el codirector del proyecto, Juan José Costa Prats
                                        perteneciente al CRAAX, ara tareas similares al director.
    \item \textbf{Experto en Inteligencia Artificial:} este proyecto constará con un experto en inteligencia artificial que dará soporte en temas más específicos que se salen fuera del alcance de
                                                    este proyecto. El experto es Jordi Nin Guerrero profesor titular en el Departamento de Operaciones, Innovación y Data Sciences en ESADE.
    \item \textbf{Colaboradores:} también contaremos con la ayuda de Xavi Verdú, profesor asociado de la UPC e investigador senior de PhD.
\end{itemize}