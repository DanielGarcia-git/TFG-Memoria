\chapter{Alcance}
\label{cap:alcance}

% Corregido

Al ser un proyecto que puede ser complejo y largo, definiremos el alcance y los objetivos que se deberán de cumplir para considerar que las hipótesis que nos
hemos planteado estén satisfechas.

\section{Objetivos}
\label{sec:objetivos}

% Corregido

Como hemos venido explicando en el capítulo \ref{cap:justificacion} nuestro principal objetivo en este proyecto es poder generar código en C compilable, realista y fiel
al original, partiendo de un fichero ejecutable en su forma de ensamblador. Para poder alcanzar este objetivo nos asistiremos con inteligencia artificial de tal manera
que podamos conseguir resultados óptimos.

Para poder alcanzar este objetivo se ha decidido dividirlo en subobjetivos, de tal manera que podamos ir dando pasos hasta el objetivo principal.
Los subobjetivos definidos son los siguientes:

\begin{enumerate}
    \item Despliegue de un modelo de redes neuronales para poder asistirnos a generar código en C a partir de un ensamblado
    \item Estudio y pruebas con modelos online tipo chatbot
\end{enumerate}

En el primer objetivo nos proponemos poder desplegar en un entorno con las capacidades técnica suficientes para poder ejecutar un modelo de red neuronal que nos
pueda asistir en las tareas de aplicar ingeniería inversa sobre un fichero ejecutable.

En el segundo objetivo nos proponemos obtener resultados preliminares que nos faciliten unos resultados orientativos para poder estimar la calidad de los resultados que podamos
obtener en nuestro objetivo principal. Para ello, nos asistiremos de modelos tipos chatbot que podemos consultar online y así poder obtener resultados inmediatos, sin necesidad
de desplegar un entorno dedicado.

\section{Posibles riesgos y obstaculos}
\label{sec:riesgos}

% Corregido

Los riesgos que se asumen en este proyecto son altos debido a la incertidumbre que hay sobre este tema. Dado que los antecedentes son prácticamente nulos y dada una tecnología
que está en auge y aun en desarrollo, los posibles problemas o resultados no esperados que nos podamos encontrar durante la investigación y el desarrollo son altos.

Alguno de los posibles obstáculos que nos podemos encontrar son:

\begin{itemize}
    \item \textbf{Resultados no deseados:} podemos encontrarnos que a loa hora de querer generar código C a partir de un ejecutable el modelo no sea capaz de procesar estos datos
                                        y, por lo tanto, nos dé una salida que no es la esperada.
    \item \textbf{Insuficiencia de recursos:} los modelos que utilizaremos son modelos muy grandes que necesita mucha memoria de GPU para poderse ejecutar, aunque sabemos que
                                        existen modelos más pequeños y que necesitan menos recursos, podemos encontrarnos de que no sea suficiente esta reducción.
\end{itemize}