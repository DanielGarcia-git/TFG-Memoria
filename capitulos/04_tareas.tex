\chapter{Descripción de tareas}
\label{cap:tareas}

% Corregido

Para la realización de este proyecto se han creado diferentes tareas para la investigación, análisis y desarrollo de este mismo. La duración del proyecto está prevista
que tenga fecha de inicio el día 18 de septiembre de 2023 y con fecha de finalización el 22 de enero de 2024. El proyecto tendrá una dedicación de trabajo de unas 450 horas,
lo correspondiente a 18 créditos ECTS\footnote{Un crédito ECTS equivale a 25 horas de trabajo del estudiante.\cite{ECTS}}.

A continuación pasaré a detallar los paquetes de tareas y las tareas que componen cada paquete.

\section{Tareas de gestión del proyecto}
\label{sec:tareas_gestion}

% Corregido

En las tareas de gestión del proyecto estarán todas las tareas que tengan que ver con gestión de proyectos, tales como definición del proyecto, presupuestos, reuniones de
sincronización, documentación relativa al proyecto, entre otras.

\subsection{Gestión del proyecto}
\label{subsec:tareas_gestion}

% Corregido

\begin{itemize}
    \item \textbf{Alcance}
        \begin{itemize}
            \item \textbf{Código:} GP01
            \item \textbf{Descripción:} Análisis y estudio del problema a estudiar, pudiendo definir el contexto y los objetivos del proyecto.
            \item \textbf{Duración:} 20h
            \item \textbf{Dependencias:} -
            \item \textbf{Recursos humanos:} Manager del proyecto
            \item \textbf{Recursos materiales:} Ordenador, Editor de texto
        \end{itemize}
    \item \textbf{Planificación}
        \begin{itemize}
            \item \textbf{Código:} GP02
            \item \textbf{Descripción:} Definición de plazos y tareas a realizar el proyecto, planificación temporal.
            \item \textbf{Duración:} 20h
            \item \textbf{Dependencias:} GP01
            \item \textbf{Recursos humanos:} Manager del proyecto
            \item \textbf{Recursos materiales:} Ordenador, Editor de texto
        \end{itemize}
    \item \textbf{Presupuesto}
        \begin{itemize}
            \item \textbf{Código:} GP03
            \item \textbf{Descripción:} Realización del coste del proyecto e informe de sostenibilidad.
            \item \textbf{Duración:} 20h
            \item \textbf{Dependencias:} GP02
            \item \textbf{Recursos humanos:} Manager del proyecto
            \item \textbf{Recursos materiales:} Ordenador, Editor de texto
        \end{itemize}
    \item \textbf{Evaluación del proyecto}
        \begin{itemize}
            \item \textbf{Código:} GP04
            \item \textbf{Descripción:} Revisar que la documentación aportada en las tareas GP01, GP02 y GP03 sea correcta y adecuada a los requisitos.
            \item \textbf{Duración:} 20h
            \item \textbf{Dependencias:} GP01, GP02, GP03
            \item \textbf{Recursos humanos:} Manager del proyecto
            \item \textbf{Recursos materiales:} Ordenador, Editor de texto
        \end{itemize}
    \item \textbf{Reuniones}
        \begin{itemize}
            \item \textbf{Código:} GP05
            \item \textbf{Descripción:} Reuniones de sincronización con el director del proyecto y con los expertos vinculados al proyecto.
            \item \textbf{Duración:} 20h
            \item \textbf{Dependencias:} - 
            \item \textbf{Recursos humanos:} Manager del proyecto, Director del proyecto
            \item \textbf{Recursos materiales:} Ordenador, Meet
        \end{itemize}
    \item \textbf{Documentación}
        \begin{itemize}
            \item \textbf{Código:} DC01
            \item \textbf{Descripción:} Creación, redactado y evaluación de la memoria de proyecto.
            \item \textbf{Duración:} 100h
            \item \textbf{Dependencias:} -
            \item \textbf{Recursos humanos:} Manager del proyecto
            \item \textbf{Recursos materiales:} Ordenador, Editor de texto
        \end{itemize}
\end{itemize}

\section{Tareas de análisis, investigación y pruebas}
\label{sec:tareas_analisis}

% Corregido

Este paquete de tareas contempla todas aquellas tareas de investigación de documentación, tecnologías, análisis y realización de pruebas sobre nuevas tecnologías
que se puedan aplicar en nuestro caso de uso. En este caso, este paquete de tareas lo dividimos en tres: las tareas propias de investigación, las tareas propias de
análisis y las tareas relacionadas con \textit{deployment} de tecnologías.

\subsection{Investigación}
\label{subsec:tareas_investigacion}

% Corregido

\begin{itemize}
    \item \textbf{Investigación sobre el concepto LLMs}
        \begin{itemize}
            \item \textbf{Código:} I01
            \item \textbf{Descripción:} Investigación sobre los \textit{Large Language Models} y los conceptos asociados a este.
            \item \textbf{Duración:} 5h
            \item \textbf{Dependencias:} -
            \item \textbf{Recursos humanos:} Desarrolladores
            \item \textbf{Recursos materiales:} Ordenador
        \end{itemize}
    \item \textbf{Investigación LLMs Open Source}
        \begin{itemize}
            \item \textbf{Código:} I02
            \item \textbf{Descripción:} Investigación sobre los LLM disponibles creados por la comunidad, valorar cuál utilizar para el proyecto.
            \item \textbf{Duración:} 5h
            \item \textbf{Dependencias:} I01
            \item \textbf{Recursos humanos:} Desarrolladores
            \item \textbf{Recursos materiales:} Ordenador
        \end{itemize}
    \item \textbf{Investigación sobre métodos de \textit{fine-tuning}}
        \begin{itemize}
            \item \textbf{Código:} I03
            \item \textbf{Descripción:} Investigación sobre que metodo de \textit{fine-tuning} se ajusta más a las necesidades del proyecto.
            \item \textbf{Duración:} 5h
            \item \textbf{Dependencias:} I01
            \item \textbf{Recursos humanos:} Desarrolladores
            \item \textbf{Recursos materiales:} Ordenador, Git
        \end{itemize}
    \item \textbf{Recogida de datos para el entrenamiento}
        \begin{itemize}
            \item \textbf{Código:} I04
            \item \textbf{Descripción:} Investigación y análisis sobre repositorios de código, con código en C que sea apto para las pruebas en nuestro modelo. 
            \item \textbf{Duración:} 5h
            \item \textbf{Dependencias:} -
            \item \textbf{Recursos humanos:} Desarrolladores
            \item \textbf{Recursos materiales:} Ordenador, Git
        \end{itemize}
\end{itemize}

\subsection{Análisis y pruebas de uso utilizando ChatGPT Plus}
\label{subsec:tareas_ichatgpt}

% Corregido

\begin{itemize}
    \item \textbf{Análisis de ChatGPT}
        \begin{itemize}
            \item \textbf{Código:} AGPT01
            \item \textbf{Descripción:} Análisis sobre la plataforma de ChatGPT, como utilizar y obtener resultados óptimos.
            \item \textbf{Duración:} 4h
            \item \textbf{Dependencias:} -
            \item \textbf{Recursos humanos:} Desarrolladores
            \item \textbf{Recursos materiales:} Ordenador, ChatGPT Plus
        \end{itemize}
    \item \textbf{Pruebas de uso con GPT-3.5}
        \begin{itemize}
            \item \textbf{Código:} AGPT02
            \item \textbf{Descripción:} Pruebas con casos de código sencillo utilizando el modelo GPT-3.5 de ChatGPT Plus.
            \item \textbf{Duración:} 4h
            \item \textbf{Dependencias:} AGPT01
            \item \textbf{Recursos humanos:} Desarrolladores
            \item \textbf{Recursos materiales:} Ordenador, ChatGPT Plus
        \end{itemize}
    \item \textbf{Pruebas de uso con GPT-4}
        \begin{itemize}
            \item \textbf{Código:} AGPT03
            \item \textbf{Descripción:} Pruebas con casos de código sencillo utilizando el modelo GPT-4 de ChatGPT Plus.
            \item \textbf{Duración:} 4h
            \item \textbf{Dependencias:} AGPT01
            \item \textbf{Recursos humanos:} Desarrolladores
            \item \textbf{Recursos materiales:} Ordenador, ChatGPT Plus
        \end{itemize}
    \item \textbf{Análisis de los resultados}
        \begin{itemize}
            \item \textbf{Código:} AGPT04
            \item \textbf{Descripción:} Análisis de los resultados obtenidos en las pruebas de uso con GPT-3.5 y GPT-4, 
                valorando si obtendremos resultados buenos con nuestro modelo.
            \item \textbf{Duración:} 4h
            \item \textbf{Dependencias:} AGPT02, AGPT03
            \item \textbf{Recursos humanos:} Desarrolladores
            \item \textbf{Recursos materiales:} Ordenador
        \end{itemize}
\end{itemize}

\subsection{Deployment de un LLM en un sistema de GPUs}
\label{subsec:tareas_gpu}

% Corregido

\begin{itemize}
    \item \textbf{Análisis y pruebas de ejecución de un LLM}
        \begin{itemize}
            \item \textbf{Código:} D01
            \item \textbf{Descripción:} Ejecución de un LLM dentro de un \textit{cluster} de GPUs obteniendo resultados básicos, las pruebas que se hagan
                no tienen que estar relacionados con nuestros casos de uso.
            \item \textbf{Duración:} 28h
            \item \textbf{Dependencias:} I
            \item \textbf{Recursos humanos:} Desarrolladores
            \item \textbf{Recursos materiales:} Ordenador, Cluster AC
        \end{itemize}
    \item \textbf{Pruebas de ejecución de nuestro target}
        \begin{itemize}
            \item \textbf{Código:} D02
            \item \textbf{Descripción:} Ejecución de casos de uso sencillos en el \textit{cluster} de GPUs.
            \item \textbf{Duración:} 12h
            \item \textbf{Dependencias:} D01
            \item \textbf{Recursos humanos:} Desarrolladores
            \item \textbf{Recursos materiales:} Ordenador, Cluster AC
        \end{itemize}
    \item \textbf{\textit{Fine-tuning}}
        \begin{itemize}
            \item \textbf{Código:} D03
            \item \textbf{Descripción:} Ejecución del \textit{fine-tuning} sobre nuestro modelo de tal manera que obtengamos mejores resultados,
                en esta tarea se ha de utilizar el \textit{data sheet}.
            \item \textbf{Duración:} 20h
            \item \textbf{Dependencias:} D02, DH01
            \item \textbf{Recursos humanos:} Desarrolladores
            \item \textbf{Recursos materiales:} Ordenador, Cluster AC, Git
        \end{itemize}
    \item \textbf{Evaluación y correción 1}
        \begin{itemize}
            \item \textbf{Código:} D04
            \item \textbf{Descripción:} En esta tarea se valorarán los resultados obtenidos de la tarea D03 y se aplicaran las correcciones que sean
                necesarias.
            \item \textbf{Duración:} 20h
            \item \textbf{Dependencias:} D03
            \item \textbf{Recursos humanos:} Desarrolladores
            \item \textbf{Recursos materiales:} Ordenador, Cluster AC, Git
        \end{itemize}
    \item \textbf{Evaluación y correción 2}
        \begin{itemize}
            \item \textbf{Código:} D05
            \item \textbf{Descripción:} En esta tarea se valorarán los resultados obtenidos de la tarea D03 y se aplicaran las correcciones que sean
                necesarias.
            \item \textbf{Duración:} 40h
            \item \textbf{Dependencias:} D04
            \item \textbf{Recursos humanos:} Desarrolladores
            \item \textbf{Recursos materiales:} Ordenador, Cluster AC, Git
        \end{itemize}
\end{itemize}

\section{Tareas de desarrollo}
\label{sec:tareas_desarrollo}

% Corregido 

En este grupo de tareas situaremos todas aquellas relacionadas con el desarrollo, ya sea el desarrollo de herramientas que ayudan a la hora del desarrollo en
sí mismo, o tareas más de creación de aplicaciones o similares.

\subsection{Desarrollo de herramientas}
\label{subsec:tareas_herramientas}

% Corregido 

\begin{itemize}
    \item \textbf{Desarrollo de scripts para la creación del Data Sheet}
        \begin{itemize}
            \item \textbf{Código:} DH01
            \item \textbf{Descripción:} Desarrollo de scripts para compilar, parsear y obtener los ensamblados. Así mismo, se creara la base de datos
                necesaria para el \textit{fine-tuning}.
            \item \textbf{Duración:} 28h
            \item \textbf{Dependencias:} -
            \item \textbf{Recursos humanos:} Desarrolladores
            \item \textbf{Recursos materiales:} Ordenador, Git
        \end{itemize}
    \item \textbf{Desarrollo de scripts de ejecución}
        \begin{itemize}
            \item \textbf{Código:} DH02
            \item \textbf{Descripción:} Desarrollo de scripts para la ejecución del LLM en el \textit{cluster} de GPUs.
            \item \textbf{Duración:} 20h
            \item \textbf{Dependencias:} -
            \item \textbf{Recursos humanos:} Desarrolladores
            \item \textbf{Recursos materiales:} Ordenador, Cluster AC, Git
        \end{itemize}
\end{itemize}

\section{Recursos}
\label{sec:recursos}
