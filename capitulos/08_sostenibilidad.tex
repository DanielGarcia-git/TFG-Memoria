\chapter{Informe de sostenibilidad}
\label{cap:sostenibilidad}

% Corregido

La sostenibilidad de un proyecto dentro de un contexto de ingeniería informática es de vital importancia, debido a que el objetivo de un ingeniero informático
es encontrar el equilibrio entre algo que es útil para la sociedad, es rentable y a la vez es sostenible, en términos ambientales, en el tiempo.

En mi caso, siempre he sido muy consciente de encontrar este equilibrio entre estos tres aspectos y sobre todo en encontrar un método sostenible medioambientalmente. Además, en
un proyecto de estas características también me ha supuesto un reto a nivel de sostenibilidad social, ya que en nuestras hipótesis planteamos diferentes
ideas que entrarían en conflicto con la actual industria del software.

Para ello, en este capítulo detallaré la sostenibilidad de este proyecto en los tres ámbitos más relevantes: el económico, el ambiental y el social.

\section{Sostenibilidad económica}
\label{sec:sostenibilidad_economica}

% Corregido

Como hemos podido ver él en capítulo \ref{cap:presupuesto} se han detallado los costes económicos de los recursos utilizados, a la hora de realizar este presupuesto se ha tenido
en cuenta que es un proyecto que actualmente despierta mucho interés dentro de la industria debido a la gran alza de las inteligencias artificiales.

Así mismo, este proyecto intenta ver el uso de las inteligencias artificiales dentro de una tarea específica, es decir, estamos atacando sobre un mercado reducido y muy específico,
pero que a la vez trae consigo mucha relevancia e importancia, dado que sus consecuencias hacen que nos planteemos la seguridad de nuestras aplicaciones a todos los niveles.

\section{Sostenibilidad ambiental}
\label{sec:sostenibilidad_ambiental}

% Corregido

Cabe destacar la importancia de la sostenibilidad ambiental en este proyecto, debido a que este, aunque no lo parezca al principio, tiene un gran impacto medioambiental, debido
a que utilizamos inteligencias artificiales que precisan de mucha potencia de cálculo, lo cual implica un gasto de energía importante, tanto para alimentar las propias máquinas
que computan estos algoritmos, como los sistemas auxiliares que ayudan a que estos sistemas funcionen.

Para que nos hagamos una idea, se estima que el entrenamiento de un PLN basado en \textit{deep learning} produce unas 284 toneladas de CO2, según la universidad de Massachusetts.
\cite{Artículo_doctrinal}. Aunque la cifra es una estimación, somos conscientes que los grandes centros de procesamiento donde se entrenan estos modelos consumen muchos recursos y, por lo tanto, a
la hora de utilizar modelos como los de ChatGPT hemos de tener en cuenta el impacto ambiental que provocamos con su uso.

En el proyecto se ha tenido en cuenta estos factores y se decidió utilizar modelos más pequeños y que consumen muchos menos recursos energéticos, ya que estos pueden ser entrenados
en tarjetas gráficas comerciales, que aunque consumen energía, es mucho menos que un gran centro de computación.

\section{Sostenibilidad social}
\label{sec:sostenibilidad_social}

% Corregido

Por lo que respecta la sostenibilidad social en el ámbito personal, este proyecto supone un reto, ya que salgo de lo que normalmente hago y me adentro en un ámbito que
para mí es desconocido, pero que a la vez me supone una gran curiosidad.

Así mismo, este proyecto supone ciertos cambios en el paradigma social, por el hecho de que hasta ahora (y si los resultados son los esperados) no hay ninguna herramienta que sea
capaz de aplicar ingeniería inversa sobre programas y generar código que sea compilable. Lo que teníamos hasta ahora son herramientas que dan una especie de pseudocódigo que
requiere de mucha intervención humana tan solo para hacerlo entendible.

Para poder evaluar si hay una necesidad real de una herramienta como la que se describe en este proyecto, me fundamento en la expectación que ha generado entre diferentes docentes, dentro y
fuera de este centro, del que han mostrado interés en saber los resultados que se extraen de este proyecto.
