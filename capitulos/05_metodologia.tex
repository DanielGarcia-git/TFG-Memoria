\chapter{Metodología y rigor}
\label{cap:metodologia}

% Corregido 31/12/2020
% TODO:daniel: Revisar la sección de metodología y rigor

En este capítulo se detallará la metodología de trabajo que se utilizará durante el desarrollo
del proyecto, así como el seguimiento que se realizará para poder evaluar el grado de cumplimiento
de los objetivos marcados.

\section{Metodología de trabajo}
\label{sec:metodologia:metodologia_trabajo}

% Corregido
% TODO:daniel: Revisar la sección de metodología de trabajo

En el desarrollo de este trabajo se utilizará una metodología ágil. En concreto, se utilizará 
la metodología SCRUM. Las metodologías ágiles son aquellas que permiten adaptar la forma de
trabajo a las condiciones del proyecto, lo cual nos permite una mayor flexibilidad e 
inmediatez en la respuesta para amoldar el proyecto y su desarrollo a las circunstancias
específicas del entorno.

En concreto, SCRUM es una metodología ágil que se basa en la división del trabajo en ciclos
llamados sprints. Estos sprints son periodos de tiempo en los que se desarrolla una parte
del proyecto. Al final de cada sprint se obtiene un producto funcional que puede ser entregado
al cliente. \cite{MetodoAgile}

Asimismo, este proyecto se dividirá en diferentes etapas que se dividirán en sprints tal y
como se detalla en el capítulo \ref{cap:tareas}. Estas son las etapas que se han definido:

\begin{itemize}
    \item \textbf{Etapa de planificación:} En esta etapa se definirán los requisitos, 
    objetivos y tareas a realizar durante el proyecto.
    \item \textbf{Etapa de investigación:} En esta etapa se investigarán los posibles 
    resultados que se pueden obtener y se definirán las herramientas que se utilizarán.
    \item \textbf{Etapa de desarrollo:} En esta etapa se desarrollará todo el software 
    necesario para el proyecto.
    \item \textbf{Etapa de pruebas:} En esta etapa se realizarán las pruebas necesarias y los 
    ajustes necesarios para obtener los resultados deseados.
\end{itemize}

Durante todas las etapas se realizarán reuniones de seguimiento con el Director del proyecto para
poder evaluar y corregir las posibles desviaciones que se puedan producir.

\section{Seguimiento}
\label{sec:metodologia:seguimiento}

% Corregido
% TODO:daniel: Revisar la sección de seguimiento

A la hora de poder evaluar, a lo largo del proyecto, el grado de cumplimiento de los objetivos
marcados, se realizarán diferentes reuniones de seguimiento que normalmente se producirán cada
dos semanas. En estas reuniones se evaluarán los objetivos cumplidos y los objetivos pendientes
de cumplir. Asimismo, se evaluarán los posibles problemas que se hayan podido producir y se
tomarán las medidas necesarias para poder corregirlos.

Las reuniones se harán de forma presencial o telemática utilizando la herramienta Google Meet
\footnote{\href{https://meet.google.com/}{Google Meet} es un servicio de videotelefonía desarrollado por Google}.

Asimismo, para el desarrollo tanto de este documento como el software que se pueda desarrollar derivado
de este proyecto se utilizará un repositorio de código fuente. En concreto, se utilizará la plataforma
GitHub \footnote{\href{https://github.com/}{GitHub} es una plataforma de desarrollo colaborativo de software
para alojar proyectos utilizando el sistema de control de versiones Git.}.

