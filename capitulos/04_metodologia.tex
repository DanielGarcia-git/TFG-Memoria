\chapter{Metodología y rigor}
\label{cap:metodologia}

\section{Metodología de trabajo}
\label{sec:metodologia:metodologia_trabajo}

En el desarrolo de este trabajo se utilizara una metodologia agil. En concreto, se utilizara 
la metodología SCRUM. as metodologías ágiles son aquellas que permiten dapatar la forma de
trabajo a las condiciones del proyecto, la cual cosa nos permite una mayor flexibilidad e 
inmediatez en la respuesta para amoldar el proyecto y su desarrollo a las circusntancias
especificas del entorno.

En concreto SCRUM es una metodología ágil que se basa en la división del trabajo en ciclos
llamados sprints. Estos sprints son periodos de tiempo en los que se desarrolla una parte
del proyecto. Al final de cada sprint se obtiene un producto funcional que puede ser entregado
al cliente. \cite{MetodoAgile}

Así mismo, este proyecto se dividira en diferentes etapas que se dividiran en sprints tal y
como se detalla en el capitulo \ref{cap:tareas}. Estas son las etapas que se han definido:

\begin{itemize}
    \item \textbf{Etapa de planificación:} En esta etapa se definiran los requisitos, 
    objetivos y tareas a realizar durante el proyecto.
    \item \textbf{Etapa de investigación:} En esta estapa se investigaran los posibles 
    resultados que se pueden obtener y se definiran las herramientas que se utilizaran.
    \item \textbf{Etapa de desarrollo:} En esta etapa se desarrollara todo el software 
    necesario para el proyecto.
    \item \textbf{Etapa de pruebas:} En esta etapa se realizaran las pruebas necesarias y los 
    ajustes necesarios para obtener los resultados deseados.
\end{itemize}

Durante todas las etapas se realizaran reuniónes de seguimiento con el Director del proyecto para
poder evaluar y correcgir las posibles desviaciones que se puedan producir.

\section{Seguimiento}
\label{sec:metodologia:seguimiento}

