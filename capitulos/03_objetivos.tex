\chapter{Objetivos}
\label{cap:objetivos}

% TODO:daniel: Añadir introducción al capitulo


\section{Identificación del problema}
\label{sec:problema}

% Corregido

Como he explicado en el capítulo \ref{cap:introducion} la ingeniería inversa es aplicada en una infinidad de campos y de los cuales hoy en día es utilizada. Pero como he mencionado
aplicar ingeniería inversa no es una tarea fácil, ya que actualmente los programas que podemos encontrar en el mercado son de tal complejidad que aplicar ingeniería inversa sobre la
totalidad del programa supone horas y horas de trabajo, teniendo el riesgo de que los resultados obtenidos no se asemejen a la realidad.

Esta complejidad no solo viene dada por el gran tamaño de los programas actuales, sino de las técnicas que se utilizan para poder ocultar aún más el programa original. Algunas
de estas técnicas son \textit{constant blinding}\footnote{Esta técnica consiste en poner un valor aleatorio a las constantes (a través de operaciones como la XOR)}, cambiar el
encoding de las variables\footnote{Busca la ocultación del valor de la variable cambiando su representación de datos}, agregación de datos\footnote{Busca agrupar variables del mismo
tipo, por ejemplo bajo un \textit{struct}}, separación de datos\footnote{Al contrario que la agregación de datos, esta busca separar los datos en unidades más pequeñas, por ejemplo
de un \textit{short} a un \textit{char}}, \textit{dead code insertion}\footnote{Agregar código redundante al programa}, \textit{loop unrolling}\footnote{Es una tecnica que aplican
los compiladores que además de disminuir el coste computacional del programa hace que el código sea menos legible}, entre otras. \cite{TecnicasIlegibleBinario}

También a la hora de aplicar ingeniería inversa debemos sumar la complejidad de las aplicaciones distribuidas, es decir, antes, cuando disponíamos de un programa disponíamos de su
código en su totalidad, aunque este fuese en forma de un binario. Con las aplicaciones distribuidas, el programa se ha segmentado en diferentes partes y estas partes cada una se
puede encontrar en una máquina diferente, provocando que no dispongamos a veces del todo el programa y, por lo tanto, de todo su diseño y lógica.

En consecuencia, debido a las técnicas de ocultación de código, las optimizaciones que el compilador puede aplicar sobre el código, el gran tamaño de los programas modernos y el auge
de las aplicaciones distribuidas hacen que la tarea de aplicar ingeniería inversa sobre un programa sea muy compleja y que las soluciones actuales, que detallaré en la sección
\ref{sec:alternativas}, no son capaces de dar buenos resultados, sino que dan una especie de pseudocódigo que nos puede ayudar a entender la lógica del programa.

Todos estos factores contribuyen a que aplicar ingeniería inversa sea muy complejo y costoso y, por lo tanto, inviable en muchos casos para ciertos escenarios.