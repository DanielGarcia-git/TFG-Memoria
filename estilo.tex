% Retoca las líneas marcadas con TODO según las necesidades

\documentclass[oneside,a4paper,12pt]{book} % TODO: cambia "oneside" por "twoside" a la hora de imprimirlo

\usepackage[spanish]{babel}
\usepackage[utf8]{inputenc}
\usepackage{geometry}
\usepackage{makeidx}
\usepackage{url}
\usepackage{graphicx}
\usepackage{color}
\usepackage{caption}
\usepackage{acronym}
\usepackage{hyphenat}
\usepackage{a4wide}
\usepackage[normalsize]{subfigure}
\usepackage{float}
\usepackage{titlesec}
\usepackage[Lenny]{fncychap}
\usepackage{listings} % para poder hacer uso de "listings" propios (p.ej. códigos)
\usepackage{eurosym} % para poder usar el símbolo del euro con \euro {xx}
\usepackage{hyperref} % TODO: añade la opción hidelinks para imprimirlo (los enlaces no aparecerán resaltados)
\usepackage{pdflscape}
\usepackage{multirow}
\usepackage[table,xcdraw]{xcolor}
\usepackage{fancyhdr}
\usepackage{longtable}
\usepackage[newfloat]{minted}
\usepackage{caption}
\usepackage{tikz-uml}
\usepackage{amsmath}
\usepackage{amssymb}
\usepackage{wrapfig}

%Fancyhdr Styles
\fancypagestyle{frontmatter}{%
    \fancyhf{} % clear all fields
    \renewcommand{\headrulewidth}{0pt}
    \lhead{}
    \lfoot[\thepage]{}
    \rfoot[]{\thepage} 
    }%
\fancypagestyle{mainmatter}{%
    \fancyhf{} % clear all fields
    \renewcommand{\headrulewidth}{0.5pt}
    \renewcommand{\footrulewidth}{0pt}
    \lhead[\leftmark]{Daniel García Estévez}
    \rhead[Daniel García Estévez]{\rightmark}
    \lfoot[\thepage]{}
    \rfoot[]{\thepage} 
}%

\setminted[]{
xleftmargin=.75cm,
xrightmargin=.75cm,
frame=single,
framesep=.25cm,
linenos,
tabsize=2,
breaklines
}

% Para que no parta las palabras
\pretolerance=10000
\renewcommand{\lstlistingname}{Algorithm}% Listing -> Algorithm
\renewcommand{\lstlistlistingname}{List of \lstlistingname s}% List of Listings -> List of Algorithms
\newcommand{\bigrule}{\titlerule[0.5mm]} \titleformat{\chapter}[display] % cambiamos el formato de los capítulos
{\bfseries\Huge} % por defecto se usaron caracteres de tamaño huge en negrita
{% contenido de la etiqueta 
\titlerule % línea horizontal 
\filright % texto alineado a la derecha 
\Large\chaptertitlename\ % capítulo e índice en tamaño large
\Large % en lugar de 
\Huge \Large\thechapter} 
{0mm} % espacio mínimo entre etiqueta y cuerpo
{\filright} % texto del cuerpo alineado a la derecha
[\vspace{0.5mm} \bigrule] % después del cuerpo, dejar espacio vertical y trazar línea horizontal gruesa
\geometry{a4paper, left=3.5cm, right=2cm, top=3cm, bottom=2cm, headsep=1.5cm}

\newenvironment{code}{\captionsetup[code]{type=listing}}{}
\SetupFloatingEnvironment{listing}{name=Código}

% Definición de mis propios tipos: Códigos, Ecuaciones y Tablas
\DeclareCaptionType{myequation}[Ecuación][Índice de ecuaciones]
\DeclareCaptionType{mycode}[Código][Índice de Códigos]

\hyphenation{fuer-tes}
\hyphenation{mul-ti-ca-pa}
\hyphenation{res-pues-ta}
\hyphenation{di-fe-ren-tes}
\hyphenation{de-sa-rro-lla-dos}
\hyphenation{re-pre-sen-tan-do}

