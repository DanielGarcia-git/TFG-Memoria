\chapter{Estrategia de entrenamiento}
\label{cap:estrategia_entrenamiento}

% Corregido 07/01/2024
% Revisado 14/01/2024

En este capítulo se explicará la estrategia de entrenamiento que se ha seguido para
entrenar los modelos de lenguaje. Para ello se explicará que modelo se ha escogido,
qué estrategias se han seguido para entrenarlo y que limitaciones nos hemos encontrado
para definir la estrategia de entrenamiento y la ejecución del entrenamiento.

\section{Elección del modelo de lenguaje}
\label{sec:modelo_lenguaje}

% Corregido 07/01/2024
% Revisado 14/01/2024

La decisión del modelo de lenguaje no ha sido una tarea fácil, debido a que podemos
encontrar una gran variedad de modelos de lenguaje en internet. Para poder escoger
el modelo de lenguaje que se va a utilizar se han tenido en cuenta los siguientes
factores:

\begin{itemize}
    \item \textbf{Tamaño:} el modelo de lenguaje debe tener un tamaño adecuado
        para poder entrenarlo en un entorno limitado.
    \item \textbf{Coste computacional:} el modelo de lenguaje debe tener un coste
        computacional adecuado para poder entrenarlo en un entorno limitado.
    \item \textbf{Especialización:} el modelo de lenguaje debe estar especializado
        en la generación y tratamiento de lenguajes de programación.
    \item \textbf{Plataforma de desarrollo:} el modelo de lenguaje debe ser compatible
        con la plataforma de desarrollo que se va a utilizar para entrenarlo.
\end{itemize}

Pero antes de entrar en detalle de los modelos de lenguaje que se han escogido,
hay que decidir bajo que plataforma interactuaremos con el modelo, es decir, bajo
que API utilizaremos para poder entrenar el modelo de lenguaje. Actualmente, encontramos
dos grandes opciones:

\begin{enumerate}
    \item \textbf{PyTorch:} es una biblioteca de código abierto para el aprendizaje
        automático. Se utiliza para aplicaciones como la visión artificial y el
        procesamiento del lenguaje natural, principalmente desarrollada por el
        grupo de investigación de inteligencia artificial de Facebook. \cite{PyTorch}
    \item \textbf{TensorFlow:} es una biblioteca de código abierto para el aprendizaje
        automático desarrollada por Google. Se utiliza para aplicaciones como la visión
        artificial y el procesamiento del lenguaje natural. \cite{TensorFlow}
\end{enumerate}

En el contexto de este proyecto, se ha decidido utilizar PyTorch. La decisión se ha
tomado debido a que utilizaremos una plataforma de desarrollo llamada Lightning AI
la cual utiliza internamente PyTorch.

Lightning AI es una plataforma de desarrollo de código abierto para la investigación
en inteligencia artificial y la producción de modelos de aprendizaje automático.
Lightning AI se basa en PyTorch y proporciona una abstracción para PyTorch. \cite{LightningAI}

Para ser más concretos utilizaremos una implementación de Lightning AI llamada
\textit{lit-gpt}\cite{litGPT} la cual nos permite entrenar e interactuar con modelos de lenguaje
de la familia GPT. En la tabla \ref{tab:litGPT} podemos ver los modelos de lenguaje
disponibles bajo esta plataforma. La ventaja de utilizar \textit{lit-gpt} es que nos permite
tener un entorno de desarrollo ya preparado para poder entrenar los modelos de lenguaje
y así poder centrarnos en la estrategia de entrenamiento que es uno de los objetivos de
este proyecto.

\begin{table}[H]
    \centering
    \resizebox{\textwidth}{!}{%
    \begin{tabular}{|l|l|l|}
        \hline
        \rowcolor[HTML]{8EA9D8} 
        \textbf{Modelo y uso} & \textbf{Tamaño del modelo} & \textbf{Referencia} \\
        \hline
        EleutherAI \href{tutorials/download_pythia.md}{Pythia} & \{14,31,70,160,410\}M, \{1,1.4,2.8,6.9,12\}B & \href{https://arxiv.org/abs/2304.01373}{Biderman et al. 2023} \\
        \hline
        LMSYS \href{tutorials/download_longchat.md}{LongChat} & 7B, 13B & \href{https://lmsys.org/blog/2023-06-29-longchat/}{LongChat Team 2023} \\
        \hline
        LMSYS \href{tutorials/download_vicuna.md}{Vicuna} & 7B, 13B, 33B & \href{https://lmsys.org/blog/2023-03-30-vicuna/}{Li et al. 2023} \\
        \hline
        Meta AI \href{tutorials/download_code_llama.md}{Code Llama} & 7B, 13B, 34B & \href{https://arxiv.org/abs/2308.12950}{Rozière et al. 2023} \\
        \hline
        Meta AI \href{tutorials/download_llama_2.md}{Llama 2} & 7B, 13B, 70B & \href{https://arxiv.org/abs/2307.09288}{Touvron et al. 2023} \\
        \hline
        Mistral AI \href{tutorials/download_mistral.md}{Mistral and Mixtral} & 7B & \href{https://mistral.ai/}{Mistral website} \\
        \hline
        Microsoft Research \href{tutorials/download_phi.md}{Phi} & 1.3B, 2.7B & \href{https://arxiv.org/abs/2309.05463}{Li et al. 2023} \\
        \hline
        NousResearch Nous-Hermes & 7B, 13B, 70B & \href{https://huggingface.co/NousResearch}{Org page} \\
        \hline
        OpenLM Research \href{tutorials/download_openllama.md}{OpenLLaMA} & 3B, 7B, 13B & \href{https://github.com/openlm-research/open_llama}{Geng \& Liu 2023} \\
        \hline
        Platypus & 7B, 13B, 70B & \href{https://arxiv.org/abs/2308.07317}{Lee, Hunter, and Ruiz 2023} \\
        \hline
        Stability AI StableCode & 3B & \href{https://stability.ai/blog/stablecode-llm-generative-ai-coding}{Stability AI 2023} \\
        \hline
        Stability AI \href{tutorials/download_freewilly_2.md}{FreeWilly2 (Stable Beluga 2)} & 70B & \href{https://stability.ai/blog/stable-beluga-large-instruction-fine-tuned-models}{Stability AI 2023} \\
        \hline
        Stability AI \href{tutorials/download_stablelm.md}{StableLM} & 3B, 7B & \href{https://github.com/Stability-AI/StableLM}{Stability AI 2023} \\
        \hline
        Stability AI \href{tutorials/download_stablelm.md}{StableLM Zephyr} & 3B & \href{https://stability.ai/blog/stablecode-llm-generative-ai-coding}{Stability AI 2023} \\
        \hline
        TII UAE \href{tutorials/download_falcon.md}{Falcon} & 7B, 40B, 180B & \href{https://falconllm.tii.ae}{TII 2023} \\
        \hline
        \href{tutorials/download_tinyllama.md}{TinyLlama} & 1.1B & \href{https://github.com/jzhang38/TinyLlama}{Zhang et al. 2023} \\
        \hline
        Together \href{tutorials/download_redpajama_incite.md}{RedPajama-INCITE} & 3B, 7B & \href{https://together.ai/blog/redpajama-models-v1}{Together 2023} \\
        \hline
        Trelis \href{tutorials/download_function_calling_llama_2.md}{Function Calling Llama 2} & 7B & \href{https://huggingface.co/Trelis/Llama-2-7b-chat-hf-function-calling-v2}{Trelis et al. 2023} \\
        \hline
    \end{tabular}%
    }
    \caption[Listado de los LLM soportados por \textit{lit-gpt}]{Listado de los LLM soportados por \textit{lit-gpt} (\cite{litGPT})}
    \label{tab:litGPT}
\end{table}

Una vez escogida la plataforma de desarrollo, se ha decidido utilizar alguno de los
modelos de lenguaje soportados por \textit{lit-gpt}. Para ello se ha tenido en cuenta los
factores que se han mencionado anteriormente. Debido a las limitaciones de hardware
que tenemos, y que se explica con más detalle en el capítulo \ref{cap:configuracion_ejecucion},
se ha decidido utilizar un modelo de lenguaje con un tamaño de tres billones de parámetros.

No se ha escogido modelos más pequeños, ya que la evolución de los LLM nos ha enseñado
que cada vez que aumentamos en número de parámetros del modelo de lenguaje, los resultados
obtenidos son mejores. Así mismo no hemos utilizado modelos más grandes, ya que el coste
computacional que supone entrenarlos es muy elevado y no disponemos de los recursos
necesarios para poder entrenarlos.

Así mismo, otra característica que se ha tenido en cuenta es que el modelo de lenguaje
esté especializado en el tratamiento de lenguajes de programación. Para ello se ha
decidido utilizar el modelo de lenguaje llamado StableCode\cite{StableCode} desarrollado
por Stability AI. Este modelo de lenguaje está especializado en el tratamiento de lenguajes
de programación. Además nos ofrece una longitud de secuencia de 16384 \textit{tokens}, lo cual
nos permite tener una mayor flexibilidad a la hora de generar el código en C y de entrenar
el modelo, ya que nuestra entrada de datos potencialmente grande.

\section{Definición de la estrategia de entrenamiento}
\label{sec:estrategia_entrenamiento}

% Corregido 09/01/2024
% Revisado 14/01/2024

Una vez escogido el modelo de lenguaje, se ha definido la estrategia de entrenamiento.
Para ello se ha tenido en cuenta los siguientes factores:

\begin{itemize}
    \item \textbf{Conjunto de entrenamiento:} el conjunto de entrenamiento debe ser
        lo más realista posible, es decir, debe contener código en ensamblador y código
        en C realista.
    \item \textbf{Longitud de secuencia:} la longitud de secuencia debe ser lo más
        grande posible, ya que nos permite tener una mayor flexibilidad a la hora de
        generar el código en C y de entrenar el modelo.
\end{itemize}

Para ello se han definido dos estrategias de entrenamiento:

\begin{enumerate}
    \item \textbf{Estrategia 1:} se construye un conjunto de entrenamiento a partir
        de código en C extraído de repositorios de GitHub y se genera el código en
        ensamblador a partir del código en C utilizando el sistema de scripts.

        Este conjunto de entrenamiento se caracteriza por tener unos datos de entrada
        muy grandes y podría comportar ciertos problemas a la hora de entrenar el modelo
        debido a la limitación de hardware (en términos de memoria VRAM) que tenemos. Así
        mismo, son casos reales y, por lo tanto, se espera que el modelo aprenda de casos
        reales y que los resultados obtenidos sean mejores.

    \item \textbf{Estrategia 2:} se construye un conjunto de entrenamiento que se compone
        de una relación entre operaciones o instrucciones básicas en C y su equivalente
        en ensamblador.

        Este conjunto de entrenamiento se caracteriza por tener unos datos de entrada
        más pequeño, es decir, la longitud de secuencia será más pequeña y, por lo tanto,
        podrá adaptarse mejor a la limitación de hardware que tenemos. Así mismo, en esta 
        estrategia se espera que el modelo aprenda a relacionar operaciones o instrucciones
        básicas en C y su equivalente en ensamblador e inferir esta traducción en contextos más
        grandes y reales.

        Cabe destacar que en esta estrategia se ha utilizado dos conjuntos de entrenamiento
        diferentes. Uno más pequeño y otro más grande, el pequeño es un subconjunto del grande.
\end{enumerate}

Para la creación del conjunto de entrenamiento de la estrategia 2 se ha utilizado un script
utilizado en el artículo \textit{Learning C to x86 Translation: An Experiment in Neural Compilation}
el cual podemos encontrar en el repositorio de GitHub con URL \url{https://github.com/jordiae/neural-compilers}.
En este artículo se crea un repositorio de GitHub el cual contiene más de un millón de funciones en C
\cite{Armengol-EstapéJordi2021LCtx}.

En la tabla \ref{tab:estrategias} podemos ver un resumen de las características de cada una de las estrategias
de entrenamiento. En concreto, el número de datos y su peso en disco, y el máximo de longitud
de secuencia.

\begin{table}[H]
    \centering
    \resizebox{\textwidth}{!}{%
    \begin{tabular}{|l|l|l|l|}
    \hline
    \rowcolor[HTML]{8EA9D8} 
    Estrategia   & Numero de entradas & Peso en MB & Secuencia máxima \\ \hline
    Estrategia 1 & 200                & 4.63       & 1091             \\ \hline
    Estrategia 2 & 99 944             & 114        & 1091             \\ \hline
    Estrategia 2 & 485 737            & 554        & 2048             \\ \hline
    \end{tabular}%
    }
    \caption[Tamaño de los \textit{training sets} utilizados]{Tamaño de los \textit{training sets} utilizados (Elaboración propia))}
    \label{tab:estrategias}
\end{table}










