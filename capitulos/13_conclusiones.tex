\chapter{Conclusiones}
\label{cap:conclusiones}

% Corregido 14/01/2024
% TODO: Revisar conclusiones

El desarrollo de este proyecto no ha sido una tarea fácil. Se trataba de un proyecto
complejo y ambicioso, pero a la vez muy interesante. El objetivo de este proyecto,
como bien he comentado en el capítulo \ref{cap:introducion}, es el de investigar
la viabilidad de utilizar modelos de lenguaje, concretamente utilizar LLM, para
realizar la ingeniería inversa de código en ensamblador. En otras palabras, investigar
si podemos asistirnos con técnicas de aprendizaje automático para mejorar los resultados
que obtenemos a la hora de aplicar ingeniería inversa a un ejecutable.

Para ello, se ha realizado un estudio e investigación de los modelos de lenguaje
disponibles gratuitamente en internet. Se ha analizado las características de cada
uno de ellos teniendo muy en cuenta su tamaño y el coste computacional que supone
entrenarlos, ya que en nuestro proyecto es una limitación que tenemos que tener muy
en cuenta. También se ha definido la estrategia de entrenamiento que se va a seguir.

También, se ha realizado un sistema de \textit{scripts} que nos permite extraer
los datos de entrenamiento de un ejecutable y generar un conjunto de entrenamiento
que se compone de un conjunto de datos de entrada, el código en ensamblador, y un
conjunto de datos de salida, el código en C.

Por lo tanto, a la pregunta de: \textbf{¿utilizando modelos de lenguaje del estilo
GPT podemos automatizar la generación de código en C a partir de un ejecutable?} Mi
respuesta es clara: \textbf{SI}. En este proyecto, a lo largo del recorrido de esta memoria
se ha demostrado que a pesar de que los resultados obtenidos no son buenos, es posible
utilizar modelos de lenguaje para automatizar la generación de código en C a partir de
un ejecutable.

Pero, ¿por qué los resultados no son buenos? La respuesta es sencilla, y es que los
modelos de lenguaje utilizados no son lo suficientemente grandes para que la habilidad
de generar código en C a partir de código en ensamblador pueda emerger. Es decir, la historia
en este aspecto es clara, cada vez que se ha aumentado el tamaño del modelo de lenguaje, es decir,
el número de parámetros que tiene los modelos, los resultados han mejorado haciendo que estos
modelos adquieran habilidades por las cuales no habían sido entrenados.

Por lo tanto, y basándome en los resultados obtenidos en el capítulo \ref{cap:viabilidad_hipotesis},
hemos visto que los resultados utilizando un modelo que tiene 600 billones de parámetros,
como es el caso de GPT-4, son muy buenos y seguramente si tuviéramos la opción de realizar un
\textit{fine-tuning} de este modelo con un conjunto de datos de entrenamiento lo suficientemente
grande y representativo de la tarea que queremos realizar, los resultados serían sorpendentes.

Basándome en los estudios que también se citan en este proyecto, concretamente en la sección
\ref{sec:estudios_previos}, se ha demostrado que los modelos de lenguaje son capaces de aprender
de realizar este tipo de tarea y que el campo de la ingeniería inversa asistida por IA es un campo
que tiene mucho futuro y aun un largo recorrido.

Así mismo, creo que las investigaciones que se realicen en este campo pueden ser muy beneficiosas
para la comunidad de la seguridad informática, y pueden ayudar a mejorar la seguridad de los sistemas
informáticos. Por ejemplo, dando un ejemplo más personal, herramientas como estas podrían ayudar dentro
de la industria de las \textit{Smart Cards}, en la cual yo trabajo, a mejorar la seguridad de este tipo
de sistemas \textit{embeded} y que son críticos en nuestra sociedad.

Por lo que respecta a si se ha logrado o no los objetivos marcados en el capítulo \ref{cap:introducion},
en este caso se ha de decir que se han cumplido los cuatro objetivos marcados. Se ha podido desplegar
un entorno de entrenamiento y se han entrenado modelos de lenguaje con diferentes tamaños de \textit{datasets}.
También se ha podido realizar pruebas con un modelo tipo \textit{chatbot}, en este caso utilizando ChatGPT
GPT-4, y se han obtenido resultados prometedores. También se ha podido generar un \textit{dataset} de entrenamiento
utilizando un sistema de \textit{scripts} que nos permite extraer los datos de entrenamiento de un ejecutable
y generar un conjunto de entrenamiento. Y por último, se ha podido realizar un estudio de los resultados obtenidos
y obtener conclusiones de estos mismos.

También se han logrado adquirir las competencias asociadas en este proyecto, concretamente la
competencia CTI3.1 se ha adquirido satisfactoriamente con la investigación que se ha realizado
y con la implementación de los \textit{scripts} que nos permiten extraer los datos de entrenamiento.
También se ha adquirido la competencia CTI3.3 con la configuración de los entornos de entrenamiento
y con la ejecución de los mismos.

En conclusión, este proyecto ha sido una experiencia muy enriquecedora y que me ha permitido
aprender mucho sobre el campo de la inteligencia artificial y de la ingeniería inversa, de
los cuales desconocía mucho. Así mismo, hemos podido poner un grano de arena en la investigación
de métodos para mejorar los procesos de ingeniería inversa y que pueden ser muy beneficiosos
para la comunidad informática.

