\chapter{Justificación}
\label{cap:justificacion}

\section{Soluciones y alternativas existentes}
\label{sec:alternativas}

Actualmente en el mercado podemos encontrar diferentes herramientas o programas llamados descompiladores o reversores los cuales tienen la capacidad de convertir un ejecutable
en un codigo C aproximado. Con aproximado me vengo a referir que el codigo que nos dan es mas parecido a un pseudocodigo muy parecido a C, pero que necesitaria ciertos ajustes
por parte nuestra para que fuese mas legible.

Así mismo, la mayoria de estas herramientas estan mas orientadas a la generación de un codigo ensamblador más "amigable" y que majoritariamente siempre necesitara intervención
para que pueda ser mas legible. Tambien añadir que estas herramientas no utilizan inteligencia artificial para mejorar sus resultados.

Algunas de estas herramientas son:

\begin{itemize}
    \item \bf IDA Pro
    \item \bf Ghidra
    \item \bf Rec Studio
    \item \bf AllyDbg
\end{itemize}

\section{Solucion tomada}
\label{sec:solucion}
