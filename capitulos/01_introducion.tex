\chapter{Introducción}
\label{cap:introducion}

\setcounter{page}{1}

\begin{flushright}
    \begin{minipage}[]{10cm}
        \emph{Sometimes, the best way to advance is in reverse}\\
    \end{minipage}\\

    Eldad Eilam, \textit{Reversing : secrets of reverse engineering}\\
\end{flushright}

\vspace{1cm}

% Corregido 29/12/2023
% TODO:daniel: Revisar la introducción

Es innegable que el \textit{software} ha cobrado una importancia en nuestra sociedad, tal es
la importancia que la industria del \textit{software} ha crecido al doble de la tasa de todas
las industrias en los últimos 10 años. Por dar un dato más concreto, en el año 2020
(el año de la pandemia) mientras la economía se contraía un 3.3\% la industria del
\textit{software} creció un 2.7\% y se espera que crezca más del doble del ritmo del PIB
mundial en los próximos cinco años. Teniendo que en septiembre de 2021 siete de las
diez empresas más valiosas del mundo son empresas de \textit{software}. \cite{IndustriaSoftware}

Es tal a importancia del \textit{software} en nuestra sociedad que muchos estados tienen planes
específicos para digitalizar su economía, como es el caso de España con su plan
España Digital 2025 o el Programa Europa Digital (DIGITAL) \cite{EspañaDigital2025}
\cite{ProgramaEuropaDigital}. Estos planes tienen como objetivo la digitalización
de la economía y la sociedad.

Pero, ¿qué es el \textit{software}? El \textit{software} es un conjunto de programas, instrucciones
y reglas informáticas que permiten ejecutar distintas tareas en una computadora.
\cite{Software} Normalmente adquirimos este \textit{software} de las empresas a través de
su formato binario. Este formato binario es el resultado de un proceso de compilación
de un lenguaje de programación de alto nivel a un lenguaje de bajo nivel, que al final
es el que entiende la máquina. Durante el proceso de compilación se pierde mucha
información, como por ejemplo los comentarios, los nombres de las variables, entre
otros.

Así mismo, las empresas también aplican técnicas de ofuscación de código, de tal
manera que el código resultante sea más difícil de entender o de intentar hacer los
pasos inversos, es decir, aplicar ingeniería inversa. Pero, ¿por qué querríamos
aplicar ingeniería inversa sobre \textit{software}? ¿Qué es la ingeniería inversa? ¿Existen
herramientas que nos ayuden a aplicar ingeniería inversa? ¿Podríamos mejorar los
resultados ayudándonos de la inteligencia artificial?

Todas estas preguntas y muchas más se intentarán responder en este proyecto de final
de grado. Para ello, se ha dividido el proyecto en diferentes capítulos, que se
detallarán a continuación:

% TODO:daniel: Revisar los capitulos de la memoria

\begin{itemize}
    \item \textbf{Capítulo 1: Introducción.} En este capítulo se introducirá el proyecto,
        se contextualizará y se detallará el contenido de esta memoria.
    \item \textbf{Capítulo 2: Estado del arte de la ingeniería inversa.} En este capítulo
        se detallará el estado del arte de la ingeniería inversa, las herramientas que
        existen actualmente y se detallarán los diferentes métodos de ingeniería inversa 
        que existen.
    \item \textbf{Capítulo 3: Justificación.} En este capítulo se detallarán el problema
        a solucionar, las alternativas existentes y la solución tomada.
    \item \textbf{Capítulo 4: Objetivos.} En este capítulo se detallarán los objetivos
        y subobjetivos que se proponen en este proyecto, así como los posibles riesgos
        y obstáculos que nos podemos encontrar durante el desarrollo del proyecto.
    \item \textbf{Capítulo 5: Metodologia y rigor.} En este capítulo se detallará la
        metodología de trabajo que se utilizará durante el proyecto. Así mismo, también
        se detallará el seguimiento que se realizará para poder evaluar el grado de
        cumplimiento de los objetivos marcados.
    \item \textbf{Capítulo 6: Planificación.} En este capítulo se detallará la planificación
        del proyecto, así como la división en tareas y la estimación de los recursos
        necesarios para poder llevar a cabo el proyecto.
    \item \textbf{Capítulo 7: Presupuesto.} En este capítulo se detallará el presupuesto
        del proyecto, así como los recursos necesarios para poder llevar a cabo el proyecto.
    \item \textbf{Capítulo 8: Sostenibilidad.} En este capítulo se detallará el impacto
        que tendrá el proyecto en el medio ambiente, así como el impacto social y económico.
    \item \textbf{Capítulo 9: Large Language Models.} En este capítulo se explicaran los
        conceptos asociados con los Large Language Models, así como los diferentes modelos
        que existen y las herramientas que se han utilizado para la relaización de este
        proyecto. Así mismo, se explicaran los posibles resultados que se puedan obtener con
        estos modelos utilizando modelos online tipo ChatGPT\footnote{ChatGPT es una aplicación
        de chatbot de inteligencia artificial desarrollado en 2022 por OpenAI que se especializa
        en el diálogo.}.
    \item \textbf{Capítulo 10: Diseño e implementación del sistema de scripts.} En este
        capítulo se detallará el diseño e implementación del sistema de scripts que se
        ha desarrollado para poder generar el \textit{dataset} que se utilizará para el
        entrenamiento del modelo.
    \item \textbf{Capítulo 11: Estrategia de entrenamiento.} En este capítulo se detallará
        la estrategia de entrenamiento que se ha utilizado para poder entrenar el modelo.
    \item \textbf{Capítulo 12: Resultados.} En este capítulo se detallarán los resultados
        obtenidos durante el entrenamiento del modelo.
    \item \textbf{Capítulo 13: Conclusiones.} En este capítulo se detallarán las conclusiones
        obtenidas durante el desarrollo del proyecto.
    \item \textbf{Capítulo 14: Trabajo futuro.} En este capítulo se detallarán los Posibles
        trabajos futuros que se pueden realizar a partir de este proyecto.
\end{itemize}

\section{Contextualización}
\label{sec:contextualizacion}

% Corregido
% Revisado 27/12/2023

Este proyecto se enmarca dentro de un proyecto de final de Grado de Ingeniería
Informática, en la especialidad de Tecnologías de la Información. Este grado se
imparte en la Facultad de Informática de Barcelona (FIB) y está dentro del contexto
de la Universitat Politècnica de Catalunya (UPC).

En este trabajo se ha propuesto abordar una serie de competencias técnicas, que
pasaré a detallar más adelante, que están estrechamente relacionadas con la
especialidad cursada. Estas son las competencias técnicas:

\begin{itemize}
    \item \textbf{CTI3.3} Diseñar, implantar y configurar redes y servicios.
    \item \textbf{CTI3.4} Diseñar \textit{software} de comunicaciones.
\end{itemize}

\section{Actores implicados}
\label{sec:actores}

% Corregido
% Revisado 29/12/2023

A continuación se pasarán a detallar a los actores implicados en el trabajo de final
de grado. Estos actores pueden participar de manera directa o indirecta dentro del
proyecto, pero tiene una relevancia importante dentro de este mismo:

\begin{itemize}
    \item \textbf{Investigador:} Este proyecto consta de un único investigador, Daniel
        García Estevez. Este se encargará de hacer tareas tales como la investigación
        de modelos a utilizar, metodos de \textit{fine-tuning}, entre otras. Así mismo,
        también se encargará del desarrollo de las herramientas necesarias para el proyecto.
    \item \textbf{Director y codirector:} El director del trabajo de final de grado, Alex
        Pajuelo Gonzalez perteneciente al CRAAX\footnote{Centre de Recerca d'Arquitectures
        Avançades de Xarxes} y profesor asociado a la UPC. Será el encargado de dirigir y
        supervisar este proyecto. Así mismo, el codirector del proyecto, Juan José Costa Prats
        perteneciente al CRAAX, ara tareas similares al director.
    \item \textbf{Experto en Inteligencia Artificial:} este proyecto constará con un experto en
        inteligencia artificial que dará soporte en temas más específicos que se salen fuera del
        alcance de este proyecto. El experto es Jordi Nin Guerrero profesor titular en el
        Departamento de Operaciones, Innovación y Data Sciences en ESADE.
    \item \textbf{Colaboradores:} también contaremos con la ayuda de Xavi Verdú, profesor asociado
        de la UPC e investigador senior de PhD.
\end{itemize}