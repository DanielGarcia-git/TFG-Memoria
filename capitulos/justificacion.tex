\chapter{Justificación}
\label{cap:justificacion}

La justificacion de este proyecto se basa en dos razones: la investigación de usos de los modelos de lenguaje basados en redes neuronales y por ultimo,
en el caso de que los resultados sean positivos, que podamos utilizar los modelos de lenguaje para poder obtener el codigo fuente de un ejecutable nos
permitiria, entre otras cosas, para el analisis de malware y seguridad, depuración y mantenimiento de software heredado, interoperabilidad, auditorias 
de software e incluso para la educación y aprendizaje.

\section{Soluciones y alternativas existentes}
\label{sec:alternativas}

Actualmente en el mercado podemos encontrar diferentes herramientas o programas llamados descompiladores o reversores los cuales tienen la capacidad de convertir un ejecutable
en un codigo C aproximado. Con aproximado me vengo a referir que el codigo que nos dan es mas parecido a un pseudocodigo muy parecido a C, pero que necesitaria ciertos ajustes
por parte nuestra para que fuese mas legible.

Así mismo, la mayoria de estas herramientas estan mas orientadas a la generación de un codigo ensamblador más "amigable" y que majoritariamente siempre necesitara intervención
para que pueda ser mas legible. Tambien añadir que estas herramientas no utilizan inteligencia artificial para mejorar sus resultados.

Algunas de estas herramientas son:

\begin{itemize}
    \item \bf IDA Pro\footnote{Enlace a la pagina web del producto \href{https://hex-rays.com/ida-pro/}{IDA Pro}}
    \item \bf Ghidra\footnote{Enlace a la pagina web del producto \href{https://ghidra-sre.org/}{Ghidra}}
    \item \bf AllyDbg\footnote{Enlace a la pagina web del producto \href{https://www.ollydbg.de/}{AllyDbg}}
\end{itemize}

\section{Solucion tomada}
\label{sec:solucion}
