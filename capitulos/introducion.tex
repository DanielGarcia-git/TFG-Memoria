\chapter{Introducción}
\label{cap:introducion}
\setcounter{page}{1}

% TODO: Descomentar cuando tengamos una frase que poner
% \begin{flushright}
%     \begin{minipage}[]{10cm}
%         \emph{Quizás algún fragmento de libro inspirador...}\\
%     \end{minipage}\\

%     Autor, \textit{Título}\\
% \end{flushright}

% \vspace{1cm}

Durante los ultimos meses hemos visto como las inteligencias artificiales han entrado de lleno en nuestras vidas. Hemos pasado de tratar estas como algo que entendian
unos pocos a abrir telediarios con inteligencias artificiales capaces de contestar cualquier pregunta que les pudiesemos hacer, resolviendo complejas tareas e incluso
planteandonos como afectara ello en nuestro sistema educativo. ¿Pero de donde salen estas inteligencias artificiales? ¿Porque pueden hacer tareas complejas? 
¿Donde esta el limite? ¿Que tipos de tareas podrian hacer?

Estas inteligencias artificiales son lo que conocemos por el nombre tecnico de Large Lenguage Model. Estos modelos cogen este nombre por su gran extensión, estamos hablando
de modelos como GPT-3 de OpenAI que tiene 175.000 millones de parametros \cite{BrownTomB2020LMaF}. Estos modelos que han sido entrenados para predecir texto, es decir, dado un texto predicen cual sera
su continuación, pero a pesar de ello han demostrado que augmentado el numero de parametros y los datos de entrenamiento son capaces de realizar tareas diferentes a les que se
le ha entrenado incialmente.

Por lo tanto, no hacen preguntar hasta donde esta el limite, que tareas son capaces de ahcer y que otras no. Nosotros nos hemos interesado por una: ¿un modelo tipo GPT-3 seria capaz
de aplicar ingieneria inversa sobre un ejecutable de tal manera que generase su codigo fuente? Entendemos que este tipo de tareas, es una tarea compleja ya que existen muchos
metodos para enborronar un ejecutable y que volver hacia atras, recuperar el codigo fuente, sea una tarea dificil o casi imposible.

Así mismo, sabiendo cuales son las limitaciones que supone un trabajo de final de grado y la premisa que nos hemos propuesto, intentaremos en este proyecto dar los primeross pasos 
para poder investigar la viabilidad de utilizar inteligencia artificial para hacer ingieneria inversa.

\section{Contextualización}
\label{sec:contextualizacion}

Como bien he mencionado con anterioridad, este proyecto se enmarca dentro de un proyecto de final de Grado de Ingieneria Informatica, en la especialidad de Tecnologias de la Información.
Este grado se imparte en la Facultat de Informatica de Barcelona (FIB) y esta dentro del contexto de la Universitat Politècnica de Catalunya (UPC).

En este trabajo se ha propuesto abordar una serie de competencias tecnicas, que pasare a detallar más adelante, que estan estrechamente relacionadas con la especialidad cursada. Estas 
son las competencias tècnicas:

\begin{itemize}
    \item \textbf{CTI3.3} Disenyar, implantar y configurar redes y servicios.
    \item \textbf{CTI3.4} Disenyar software de comunicaciones.
\end{itemize}

\section{Indetificación del problema}
\label{sec:problema}

El proyecto busca atajar el problema que actualmente tenemos a la hora de aplicar ingieneria inversa sobre ejecutables. Actuamente, es una tarea compleja debido a que las optimizaciónes
realizadas en tiempo de compilación hace que un codigo en C pierda totalmente la estructura en un codig en ensamblador. Así mismo, la falta de una tabla de simbolos o información relevante
al nombramiento de funciones o variables hace que el codigo sea menos legible.

Tambien sabemos, que todas estas modificaciones que se hace en un codigo compilado siguen ciertos patrones, ya que afinal aplicamos metodos deterministas para hacer estas optimizaciones 
o ocultaciones de codigo. Debido a estos patrones, creemos que con una cantidad de datos representativo y con la asistencia de inteligencia artificial podamos acelerar este proceso.

Por lo tanto, lo que se pretende en esta investigación es observar si con la ayuda de inteligencia artificial, concretamente a atraves de modelos del lenguaje, podemos obtener mejores 
resultados que con las herramientas que hay actualmente en el mercado.

\section{Terminologia y definiciones}
\label{sec:terminalogia}

En este apartado pasaremos a detallar ciertos terminos reiterativos que veremos a lo largo del proyecto y que son de especial importancia. Las definiciones son las siguientes:

\begin{itemize}
    \item \textbf{Large Lenguage Model:} un LLM o \textit{Large Lenguage Model} son modelos \textit{Pre-trained Lenguage Models}\footnote{El concepto de preentrenamiento en un modelo 
                                         de lenguaje está relacionado con el aprendizaje por transferencia. La idea del aprendizaje por transferencia es reutilizar los conocimientos 
                                         aprendidos en una o varias tareas y aplicarlos a tareas nuevas.} (PLM) a los cuales se han augmentado o el tamaño del modelo en si o los datos.
                                         Con este augmento se dieron cuenta que havia una notoria mejora en terminos de \textit{performance} y de la capcidad del modelos en hacer 
                                         ciertas tareas. \cite{ZhaoWayneXin2023ASoL}
    \item \textbf{Fine-tuning:} es una técnica de entrenamiento que consiste en reutilizar un modelos predefinido y preentrenado, de tal manera que ajustamos ciertas capas de la 
                                red neuronal para obtener mejores resultados para nuestra tarea en especifico.
\end{itemize}

\section{Actores implicados}
\label{sec:terminalogia}

A continuación se pasaran a detallar a los actores implicados en el trabajo. Estos actores pueden participar de manera activa o pasiva dentro del proyecto, pero tiene una relevancia 
importante dentro de este mismo:

\begin{itemize}
    \item \textbf{Investigador:} Este proyecto consta de un unico investigador, Daniel García Estevez. Este se encargara de hacer tareas tales como la investigación de modelos a 
                                 a utilizar, metodos de \textit{fine-tuning}, entre otras.
    \item \textbf{Director y codirector:} El director del trabajo de final de grado Alex Pajuelo Gonzalez perteneciente al CRAAX\footnote{Centre de Recerca d'Arquitectures Avançades de Xarxes}
                                          y profesor asociado a la UPC. Sera el encargado de dirigir y supervisar este proyecto. Así mismo, el codirector del proyecto Juan José Costa Prats 
                                          perteneciente al CRAAX, ara tareas similares al director.
    \item \textbf{Experto en Inteligencia Artificial:} este proyecto constara con un experto en inteligencia artificial que dara soporte en temas más especificos que se salen fuera del alcance de
                                                       este proyecto. El experto es Jordi Nin Guerrero profesor titular en el Departamento de Operaciones, Innovación y Data Sciences en ESADE.
    \item \textbf{Colaboradores:} tambien contaremos con la ayuda de Xavi Verdú, profesor asociado de la UPC y investigador senior de PhD.
\end{itemize}