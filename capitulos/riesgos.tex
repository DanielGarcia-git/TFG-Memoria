\chapter{Gestión de riesgos: Planes alternativos y obstáculos}
\label{cap:riesgos}

% Corregido

A continuación, como se menciona en la sección \textbf{Posibles riesgos y obstaculos} \ref{}, donde definimos los riegos
que supone un proyecto de este tipo, detallaremos los planes alternativos para solucionar los riegos a los cuales nos enfrentamos.
Así mismo, se detallará la probabilidad y el impacto que un riesgo pueda tener sobre nuestro proyecto.

\begin{table}[H]
    \centering
    \resizebox{\textwidth}{!}{%
    \begin{tabular}{|l|l|l|c|}
    \hline
    \rowcolor[HTML]{8EA9D8} 
    Riesgos                                                                               & Probabilidad & Impacto & \multicolumn{1}{l|}{\cellcolor[HTML]{8EA9D8}Plan alternativo}                                                                                                                                                                                                                                                                                                                                                                          \\ \hline
    Fecha fija de entrega                                                                 & Baja         & Bajo    & \begin{tabular}[c]{@{}c@{}}En los proyectos con deadlines ajustados hemos de considerar la fecha de\\ entrega como un riesgo, pero al utilizar un sistema agile, esto nos permite\\ en cada iteración poder corregir todos los desvíos o imprevistos que puedan\\ surgir\end{tabular}                                                                                                                                                  \\ \hline
    \begin{tabular}[c]{@{}l@{}}Inesperiencia en las\\ tecnologias utilizadas\end{tabular} & Baja         & Medio   & \begin{tabular}[c]{@{}c@{}}Como el objetivo del proyecto no es centrarse en los modelos de lenguaje\\ si no, en utilizarlos como medio para obtener un resultado, el plan alternativo\\ pasa por no dar prioridad al medio sino al fin\end{tabular}                                                                                                                                                                                    \\ \hline
    \begin{tabular}[c]{@{}l@{}}Resultados obtenidos\\ no esperados\end{tabular}           & Media        & Alto    & \begin{tabular}[c]{@{}c@{}}En proyectos de investigación la incertidumbre es alta, ya que no sabemos con\\ total seguridad los resultados que obtendremos, por lo tanto, puede ser que no \\ sean los esperados. En consecuencia, en caso de que los resultados no sean\\ los esperados o no sean de la calidad esperada, el plan alternativo será intentar\\ modificar los métodos utilizados para afinar los resultados\end{tabular} \\ \hline
    \end{tabular}%
    }
    \caption{Identificación de riesgos y planes alternativos.}
    \label{tab:riegos}
\end{table}