\chapter{Presupuesto}
\label{cap:presupuesto}

\section{Identificación y estimación de gastos}
\label{sec:identificacion_gastos}

% Corregido

Para el desarrollo del proyecto es necesario una serie de gastos asociados a diferentes tipos de recursos. En estos recursos podemos identificar los
recursos humanos, los recursos materiales tanto directos como indirectos.

Los gastos en recursos humanos son aquellos gastos relacionados con el pago de salarios por el trabajo que realizaran cada actor implicado en el proyecto.
Los gastos en recursos materiales encontramos de dos tipos: los directos, que son aquellos que influyen directamente en el proyecto y que son necesarios, como
el equipamiento informático. Y los recursos materiales indirectos son recursos que no influyen directamente sobre el proyecto, pero se han de tener en cuenta,
estos recursos son tales como el alquiler de un espacio de trabajo, por ejemplo.

Así mismo, como en todos los proyectos, sabemos que asimilamos ciertos riesgos e imprevistos y, por lo tanto, en el presupuesto se han de tomar en consideración y
reservar una parte para contingencias. En las próximas secciones detallaré el presupuesto por cada tipo de recurso.

\section{Recursos humanos}
\label{sec:recursos_humanos}

% Corregido

Para poder definir el presupuesto dedicado a los recursos humanos deberemos de identificar los actores implicados en el proceso y su sueldo medio. Para poder definir que sueldo
le asignamos a cada actor implicado hemos utilizado una página web llamada \textit{Glassdoor}\footnote{Glassdoor es un sitio web estadounidense en el que empleados actuales y
antiguos hacen reseñas anónimas de empresas. \cite{Glassdoor}}.

En nuestro proyecto, tal y como definimos en la sección \ref{sec:actores}, los actores implicados son: el responsable de proyectos de inteligencia artificial e innovación,
el investigador de inteligencia artificial y un programador junior con conocimientos en \textit{scripting}. En la tabla \ref{tab:presupuesto_roles} podremos observar el precio
por hora de cada actor y el precio por hora añadiendo los impuestos necesarios. Para el cálculo de impuestos hemos estimado que de media una empresa paga un 30\% de impuestos por
cada trabajador.

\begin{table}[H]
    \centering
    \resizebox{\textwidth}{!}{%
    \begin{tabular}{|l|c|c|}
    \hline
    \rowcolor[HTML]{8EA9D8} 
    \textbf{Rol}                                                                                               & \multicolumn{1}{l|}{\cellcolor[HTML]{8EA9D8}\textbf{Sueldo neto (hora)}} & \multicolumn{1}{l|}{\cellcolor[HTML]{8EA9D8}\textbf{Sueldo neto + impuestos (hora)}} \\ \hline
    \begin{tabular}[c]{@{}l@{}}Responsable de proyectos de\\ inteligencia artificial e innovación\end{tabular} & 30.42 €                                                                  & 39.55 €                                                                              \\ \hline
    Investigador inteligencia artificial                                                                       & 27.13 €                                                                  & 35.27 €                                                                              \\ \hline
    Programador junior                                                                                         & 10.83 €                                                                  & 14.08 €                                                                              \\ \hline
    \end{tabular}%
    }
    \caption{Salarios netos medios por hora para cada rol (Glassdor)}
    \label{tab:presupuesto_roles}
\end{table}

A continuación detallaré lo que tendremos que pagar en concepto de sueldos a cada trabajador por las tareas que hace cada uno y la dedicación estimada. Para ello, utilizaremos
las tareas que definimos en el capítulo \ref{cap:tareas}. Detallaremos por cada tarea quién es el actor implicado, el precio por hora de esa tarea, que viene determinado
por los actores implicados, los impuestos por hora y el total de sueldo neto e impuestos. Así mismo, también detallaré el total por cada paquete de tareas.

En la tabla \ref{tab:tareas_presupuesto} podréis ver esta información, cabe destacar que en esta apartado no podréis ver reflejado las contingencias, ya que esto
lo detallaremos en la sección \ref{sec:contingencias}

\begin{table}[H]
    \centering
    \resizebox{\textwidth}{!}{%
    \begin{tabular}{|llll|lll|}
    \hline
    \rowcolor[HTML]{8EA9D8} 
    \multicolumn{1}{|c|}{\cellcolor[HTML]{8EA9D8}\textbf{Código tarea}} & \multicolumn{1}{c|}{\cellcolor[HTML]{8EA9D8}\textbf{Número de horas}} & \multicolumn{1}{c|}{\cellcolor[HTML]{8EA9D8}\textbf{Rol}} & \multicolumn{1}{c|}{\cellcolor[HTML]{8EA9D8}\textbf{Precio por hora (neto)}} & \multicolumn{1}{c|}{\cellcolor[HTML]{8EA9D8}\textbf{Total neto}} & \multicolumn{1}{c|}{\cellcolor[HTML]{8EA9D8}\textbf{Impuestos por hora}}      & \multicolumn{1}{c|}{\cellcolor[HTML]{8EA9D8}\textbf{Total impuestos}} \\ \hline
    \multicolumn{1}{|l|}{GP01}                                          & \multicolumn{1}{l|}{20}                                               & \multicolumn{1}{l|}{Responsable de proyectos}             & 30.42 €                                                                      & \multicolumn{1}{l|}{608.4 €}                                     & \multicolumn{1}{l|}{9.13 €}                                                   & 182.52 €                                                              \\ \hline
    \multicolumn{1}{|l|}{GP02}                                          & \multicolumn{1}{l|}{20}                                               & \multicolumn{1}{l|}{Responsable de proyectos}             & 30.42 €                                                                      & \multicolumn{1}{l|}{608.4 €}                                     & \multicolumn{1}{l|}{9.13 €}                                                   & 182.52 €                                                              \\ \hline
    \multicolumn{1}{|l|}{GP03}                                          & \multicolumn{1}{l|}{20}                                               & \multicolumn{1}{l|}{Responsable de proyectos}             & 30.42 €                                                                      & \multicolumn{1}{l|}{608.4 €}                                     & \multicolumn{1}{l|}{9.13 €}                                                   & 182.52 €                                                              \\ \hline
    \multicolumn{1}{|l|}{GP04}                                          & \multicolumn{1}{l|}{20}                                               & \multicolumn{1}{l|}{Responsable de proyectos}             & 30.42 €                                                                      & \multicolumn{1}{l|}{608.4 €}                                     & \multicolumn{1}{l|}{9.13 €}                                                   & 182.52 €                                                              \\ \hline
    \multicolumn{1}{|l|}{GP05}                                          & \multicolumn{1}{l|}{20}                                               & \multicolumn{1}{l|}{Responsable de proyectos}             & 30.42 €                                                                      & \multicolumn{1}{l|}{608.4 €}                                     & \multicolumn{1}{l|}{9.13 €}                                                   & 182.52 €                                                              \\ \hline
    \multicolumn{1}{|l|}{DC01}                                          & \multicolumn{1}{l|}{176}                                              & \multicolumn{1}{l|}{Responsable de proyectos}             & 30.42 €                                                                      & \multicolumn{1}{l|}{5353.92 €}                                   & \multicolumn{1}{l|}{9.13 €}                                                   & 1606.78 €                                                             \\ \hline
    \rowcolor[HTML]{A4BAE0} 
    \multicolumn{4}{|l|}{\cellcolor[HTML]{A4BAE0}\textbf{Total paquete neto}}                                                                                                                                                                                                              & \multicolumn{1}{l|}{\cellcolor[HTML]{A4BAE0}\textbf{8395.92 €}}  & \multicolumn{1}{l|}{\cellcolor[HTML]{A4BAE0}\textbf{Total paquete impuestos}} & \textbf{2518.78 €}                                                    \\ \hline
    \multicolumn{1}{|l|}{I01}                                           & \multicolumn{1}{l|}{5}                                                & \multicolumn{1}{l|}{Investigador en IA}                   & 27.14 €                                                                      & \multicolumn{1}{l|}{135.7 €}                                     & \multicolumn{1}{l|}{8.14 €}                                                   & 40.71 €                                                               \\ \hline
    \multicolumn{1}{|l|}{I02}                                           & \multicolumn{1}{l|}{5}                                                & \multicolumn{1}{l|}{Investigador en IA}                   & 27.14 €                                                                      & \multicolumn{1}{l|}{135.7 €}                                     & \multicolumn{1}{l|}{8.14 €}                                                   & 40.71 €                                                               \\ \hline
    \multicolumn{1}{|l|}{I03}                                           & \multicolumn{1}{l|}{5}                                                & \multicolumn{1}{l|}{Investigador en IA}                   & 27.14 €                                                                      & \multicolumn{1}{l|}{135.7 €}                                     & \multicolumn{1}{l|}{8.14 €}                                                   & 40.71 €                                                               \\ \hline
    \multicolumn{1}{|l|}{I04}                                           & \multicolumn{1}{l|}{5}                                                & \multicolumn{1}{l|}{Investigador en IA}                   & 27.14 €                                                                      & \multicolumn{1}{l|}{135.7 €}                                     & \multicolumn{1}{l|}{8.14 €}                                                   & 40.71 €                                                               \\ \hline
    \rowcolor[HTML]{A4BAE0} 
    \multicolumn{4}{|l|}{\cellcolor[HTML]{A4BAE0}\textbf{Total paquete neto}}                                                                                                                                                                                                              & \multicolumn{1}{l|}{\cellcolor[HTML]{A4BAE0}\textbf{542.8 €}}    & \multicolumn{1}{l|}{\cellcolor[HTML]{A4BAE0}\textbf{Total paquete impuestos}} & \textbf{162.84 €}                                                     \\ \hline
    \multicolumn{1}{|l|}{AGPT01}                                        & \multicolumn{1}{l|}{4}                                                & \multicolumn{1}{l|}{Investigador en IA}                   & 27.14 €                                                                      & \multicolumn{1}{l|}{108.56 €}                                    & \multicolumn{1}{l|}{8.14 €}                                                   & 32.57 €                                                               \\ \hline
    \multicolumn{1}{|l|}{AGPT02}                                        & \multicolumn{1}{l|}{4}                                                & \multicolumn{1}{l|}{Investigador en IA}                   & 27.14 €                                                                      & \multicolumn{1}{l|}{108.56 €}                                    & \multicolumn{1}{l|}{8.14 €}                                                   & 32.57 €                                                               \\ \hline
    \multicolumn{1}{|l|}{AGPT03}                                        & \multicolumn{1}{l|}{4}                                                & \multicolumn{1}{l|}{Investigador en IA}                   & 27.14 €                                                                      & \multicolumn{1}{l|}{108.56 €}                                    & \multicolumn{1}{l|}{8.14 €}                                                   & 32.57 €                                                               \\ \hline
    \multicolumn{1}{|l|}{AGPT04}                                        & \multicolumn{1}{l|}{4}                                                & \multicolumn{1}{l|}{Investigador en IA}                   & 27.14 €                                                                      & \multicolumn{1}{l|}{108.56 €}                                    & \multicolumn{1}{l|}{8.14 €}                                                   & 32.57 €                                                               \\ \hline
    \rowcolor[HTML]{A4BAE0} 
    \multicolumn{4}{|l|}{\cellcolor[HTML]{A4BAE0}\textbf{Total paquete neto}}                                                                                                                                                                                                              & \multicolumn{1}{l|}{\cellcolor[HTML]{A4BAE0}\textbf{434.24 €}}   & \multicolumn{1}{l|}{\cellcolor[HTML]{A4BAE0}\textbf{Total paquete impuestos}} & \textbf{130.27 €}                                                     \\ \hline
    \multicolumn{1}{|l|}{D01}                                           & \multicolumn{1}{l|}{28}                                               & \multicolumn{1}{l|}{Investigador en IA}                   & 27.14 €                                                                      & \multicolumn{1}{l|}{759.92 €}                                    & \multicolumn{1}{l|}{8.14 €}                                                   & 227.98 €                                                              \\ \hline
    \multicolumn{1}{|l|}{D02}                                           & \multicolumn{1}{l|}{12}                                               & \multicolumn{1}{l|}{Investigador en IA}                   & 27.14 €                                                                      & \multicolumn{1}{l|}{325.68 €}                                    & \multicolumn{1}{l|}{8.14 €}                                                   & 97.70 €                                                               \\ \hline
    \multicolumn{1}{|l|}{D03}                                           & \multicolumn{1}{l|}{20}                                               & \multicolumn{1}{l|}{Investigador en IA}                   & 27.14 €                                                                      & \multicolumn{1}{l|}{542.8 €}                                     & \multicolumn{1}{l|}{8.14 €}                                                   & 162.84 €                                                              \\ \hline
    \multicolumn{1}{|l|}{D04}                                           & \multicolumn{1}{l|}{20}                                               & \multicolumn{1}{l|}{Investigador en IA}                   & 27.14 €                                                                      & \multicolumn{1}{l|}{542.8 €}                                     & \multicolumn{1}{l|}{8.14 €}                                                   & 162.84 €                                                              \\ \hline
    \multicolumn{1}{|l|}{D05}                                           & \multicolumn{1}{l|}{40}                                               & \multicolumn{1}{l|}{Investigador en IA}                   & 27.14 €                                                                      & \multicolumn{1}{l|}{1085.6 €}                                    & \multicolumn{1}{l|}{8.14 €}                                                   & 325.68 €                                                              \\ \hline
    \rowcolor[HTML]{A4BAE0} 
    \multicolumn{4}{|l|}{\cellcolor[HTML]{A4BAE0}\textbf{Total paquete neto}}                                                                                                                                                                                                              & \multicolumn{1}{l|}{\cellcolor[HTML]{A4BAE0}\textbf{3256.8 €}}   & \multicolumn{1}{l|}{\cellcolor[HTML]{A4BAE0}\textbf{Total paquete impuestos}} & \textbf{977.04 €}                                                     \\ \hline
    \multicolumn{1}{|l|}{DH01}                                          & \multicolumn{1}{l|}{28}                                               & \multicolumn{1}{l|}{Programador junior}                   & 10.83 €                                                                      & \multicolumn{1}{l|}{303.24 €}                                    & \multicolumn{1}{l|}{3.25 €}                                                   & 90.97 €                                                               \\ \hline
    \multicolumn{1}{|l|}{DH02}                                          & \multicolumn{1}{l|}{18}                                               & \multicolumn{1}{l|}{Programador junior}                   & 10.83 €                                                                      & \multicolumn{1}{l|}{194.94 €}                                    & \multicolumn{1}{l|}{3.25 €}                                                   & 58.48 €                                                               \\ \hline
    \rowcolor[HTML]{A4BAE0} 
    \multicolumn{4}{|l|}{\cellcolor[HTML]{A4BAE0}\textbf{Total paquete neto}}                                                                                                                                                                                                              & \multicolumn{1}{l|}{\cellcolor[HTML]{A4BAE0}\textbf{498.18 €}}   & \multicolumn{1}{l|}{\cellcolor[HTML]{A4BAE0}\textbf{Total paquete impuestos}} & \textbf{149.45 €}                                                     \\ \hline
    \rowcolor[HTML]{A4BAE0} 
    \multicolumn{4}{|l|}{\cellcolor[HTML]{A4BAE0}\textbf{Total neto}}                                                                                                                                                                                                                      & \multicolumn{3}{c|}{\cellcolor[HTML]{A4BAE0}\textbf{13127.94 €}}                                                                                                                                                         \\ \hline
    \rowcolor[HTML]{A4BAE0} 
    \multicolumn{4}{|l|}{\cellcolor[HTML]{A4BAE0}\textbf{Total impuestos}}                                                                                                                                                                                                                 & \multicolumn{3}{c|}{\cellcolor[HTML]{A4BAE0}\textbf{3938.38 €}}                                                                                                                                                          \\ \hline
    \rowcolor[HTML]{A4BAE0} 
    \multicolumn{4}{|l|}{\cellcolor[HTML]{A4BAE0}\textbf{Total (neto + impuestos)}}                                                                                                                                                                                                        & \multicolumn{3}{c|}{\cellcolor[HTML]{A4BAE0}\textbf{17066.32 €}}                                                                                                                                                         \\ \hline
    \end{tabular}%
    }
    \caption{Detalle de gastos por personal por cada tarea (Elaboración propia)}
    \label{tab:tareas_presupuesto}
\end{table}

\section{Recursos materiales directos}
\label{sec:recursos_directos}

% Corregido

Tal y como hemos definido en la sección \ref{sec:identificacion_gastos}, los recursos materiales directos son aquellos que influyen directamente sobre el
proyecto. En este caso, disponemos de 3 recursos materiales directos. Un portátil, una torre y un servicio de computación en la nube donde disponemos de 8 GPUs.

Como se puede apreciar en la tabla \ref{tab:amortizacion} el servicio de computación en la nube no tiene ningún gasto. Esto se debe a que es un servicio que ofrece
gratuitamente el departamento de Arquitectura de Computadores, de la Facultad de Informática de Barcelona, perteneciente a la Universidad Politécnica de Cataluña.
Este servicio nos ofrece acceso a 8 GPU NVIDIA RTX 2080TI 11 GB GDDR6 PCIe\footnote{Información disponible en \href{https://www.ac.upc.edu/ca/nosaltres/serveis-tic/blog/nou-node-amb-gpus-al-cluster-sert}{Departamento de Arquitectura de Computadores}},
que utilizaremos para ejecutar nuestros modelos de lenguaje.

Así mismo, en la tabla \ref{tab:amortizacion} podréis observar que debido a que los recursos son finitos, es decir, tienen una vida útil finita, se ha calculado la
amortización asociada siguiendo la siguiente fórmula:

\begin{myequation}[h]
    \begin{equation}
        Amortizacion= \frac{Precio}{Vida\ util \cdot 249\cdot Dedicacion}\cdot Duracion\ proyecto
    \label{ec:ec1}
    \end{equation}
    \caption{Equación amortización de los recursos materiales}
\end{myequation} 

\begin{table}[H]
    \centering
    \resizebox{\textwidth}{!}{%
    \begin{tabular}{|llcl|l|}
    \hline
    \rowcolor[HTML]{8EA9D8} 
    \multicolumn{1}{|c|}{\cellcolor[HTML]{8EA9D8}\textbf{Recursos materiales}} & \multicolumn{1}{c|}{\cellcolor[HTML]{8EA9D8}\textbf{Precio}} & \multicolumn{1}{c|}{\cellcolor[HTML]{8EA9D8}\textbf{Cantidad}} & \multicolumn{1}{c|}{\cellcolor[HTML]{8EA9D8}\textbf{Vida útil (años)}} & \multicolumn{1}{c|}{\cellcolor[HTML]{8EA9D8}\textbf{Amortización}} \\ \hline
    \multicolumn{1}{|l|}{LG Gram 15Z980-BAA}                                   & \multicolumn{1}{l|}{1040.90 €}                               & \multicolumn{1}{c|}{1}                                         & 4                                                                      & 117.57 €                                                           \\ \hline
    \multicolumn{1}{|l|}{Ordenador sobremesa custom}                           & \multicolumn{1}{l|}{500 €}                                   & \multicolumn{1}{c|}{1}                                         & 4                                                                      & 56.48 €                                                            \\ \hline
    \multicolumn{1}{|l|}{Servicio computación en la nube}                      & \multicolumn{1}{l|}{0 €}                                     & \multicolumn{1}{c|}{1}                                         & -                                                                      & No Amortizable                                                     \\ \hline
    \multicolumn{4}{|r|}{\textbf{TOTAL}}                                                                                                                                                                                                                                                & 174.05 €                                                           \\ \hline
    \end{tabular}%
    }
    \caption{Amortización de los recursos materiales directos (Elaboración propia)}
    \label{tab:amortizacion}
\end{table}

\section{Recursos materiales inderectos}
\label{sec:recursos_indirectos}

% Corregido

Tal y como he explicado en la sección \ref{sec:identificacion_gastos} en un proyecto de estas características también disponemos de recursos materiales indirectos, que son aquellos
que no influyen directamente en el proyecto, pero que se han de tener en cuenta.

En este proyecto hemos contemplado un único recurso indirecto que son el espacio de trabajo donde se puede desarrollar este proyecto. En este caso, al ser un proyecto
temporal y con la única necesidad de tener conexión a internet, se ha decidió por utilizar un espacio de \textit{co-working}\footnote{El cotrabajo o trabajo en oficina integrada
(del inglés coworking) es una forma de trabajo que permite a profesionales independientes, emprendedores, y pymes de diferentes sectores, compartir un mismo espacio de trabajo,
tanto físico como virtual, para desarrollar sus proyectos profesionales de manera independiente.\cite{CoWorking}}

\begin{table}[H]
    \centering
    \begin{tabular}{llll|l|}
    \cline{2-5}
    \multicolumn{1}{l|}{}                   & \multicolumn{1}{c|}{\cellcolor[HTML]{8EA9D8}\textbf{Precio mensual}} & \multicolumn{1}{c|}{\cellcolor[HTML]{8EA9D8}\textbf{IVA (21\%)}} & \multicolumn{1}{c|}{\cellcolor[HTML]{8EA9D8}\textbf{Meses}} & \multicolumn{1}{c|}{\cellcolor[HTML]{8EA9D8}\textbf{Total}} \\ \hline
    \multicolumn{1}{|l|}{Espacio coworking} & \multicolumn{1}{l|}{1640 €}                                          & \multicolumn{1}{l|}{1984.4 €}                                   & 4                                                           & 7937.6 €                                                    \\ \hline
    \multicolumn{4}{|r|}{\textbf{TOTAL}}                                                                                                                                                                                                           & 7937.6 €                                                    \\ \hline
    \end{tabular}
    \caption{Coste de los recursos materiales indirectos (Elaboración propia)}
    \label{tab:oficina}
\end{table}

Como podemos observar en la tabla \ref{tab:oficina} tenemos que el espacio de \textit{co-working} es una cuota mensual, donde en esta misma se incluye los gastos derivados de
electricidad, internet, agua, entre otros gastos. Al tratarse de un bien que no poseemos, este no será amortizable.

\section{Contingencias}
\label{sec:contingencias}

% Corregido

Al tratarse de un proyecto de investigación y por la naturaleza de estos mismos, la incertidumbre es alta y, por lo tanto, las contingencias se llevarán un
porcentaje del presupuesto alto, debido a que deberemos ser más cautelosos.

Para poder estimar las contingencias se han tenido dos imprevistos en cuenta que podremos ver reflejados en la tabla \ref{tab:contingencias}. Para calcular el
precio de estos imprevistos, hemos tenido en cuenta el número de horas que se deberían de hacer más para poder contener este imprevisto.

En el caso del imprevisto sea la inexperiencia, el número de horas que podrían hacerse serán del orden de 50 horas más de trabajo para el investigador de inteligencia artificial.
Por lo que respecta al imprevisto de que los resultados no sean los esperados, se ha calculado que se necesitaran unas 60 horas de investigación extra para poder
recalcular las conclusiones.

\begin{table}[H]
    \centering
    \begin{tabular}{|lll|l|}
    \hline
    \rowcolor[HTML]{8EA9D8} 
    \multicolumn{1}{|c|}{\cellcolor[HTML]{8EA9D8}\textbf{Imprevisto}} & \multicolumn{1}{c|}{\cellcolor[HTML]{8EA9D8}\textbf{Probabilidad}} & \multicolumn{1}{c|}{\cellcolor[HTML]{8EA9D8}\textbf{Precio}} & \multicolumn{1}{c|}{\cellcolor[HTML]{8EA9D8}\textbf{Coste}} \\ \hline
    \multicolumn{1}{|l|}{Inexperiencia en las tecnologías utilizadas} & \multicolumn{1}{l|}{50 \%}                                          & 1763.5 €                                                     & 881.75 €                                                    \\ \hline
    \multicolumn{1}{|l|}{Resultados obtenidos no esperados}           & \multicolumn{1}{l|}{60 \%}                                          & 2116.2 €                                                     & 1269.72 €                                                   \\ \hline
    \multicolumn{3}{|r|}{\textbf{TOTAL}}                                                                                                                                                                  & 2151.47 €                                                   \\ \hline
    \end{tabular}
    \caption{Coste de los imprevistos (Elaboración propia)}
    \label{tab:contingencias}
\end{table}

\section{Presupuesto final}
\label{sec:presupuesto_final}

Una vez analizadas todas las secciones anteriores, obtenemos en la tabla \ref{tab:presupuesto_final} el total del presupuesto del proyecto con contingencias (las cuales suponen el 
7.87 \% del presupuesto total) y los impuestos oportunos.

\begin{table}[H]
    \centering
    \begin{tabular}{|l|l|}
    \hline
    \rowcolor[HTML]{8EA9D8} 
    \multicolumn{1}{|c|}{\cellcolor[HTML]{8EA9D8}\textbf{Resumen}} & \multicolumn{1}{c|}{\cellcolor[HTML]{8EA9D8}\textbf{Precio}} \\ \hline
    Recursos humanos                                               & 17066.32 €                                                   \\ \hline
    Recursos materiales directos                                   & 174.05 €                                                     \\ \hline
    Recursos materiales indirectos                                 & 7937.9 €                                                     \\ \hline
    Contingencias                                                  & 2151.47 €                                                    \\ \hline
    \multicolumn{1}{|r|}{\textbf{TOTAL}}                           & 27329.74 €                                                   \\ \hline
    \end{tabular}
    \caption{Presupuesto final (Elaboración propia)}
    \label{tab:presupuesto_final}
\end{table}