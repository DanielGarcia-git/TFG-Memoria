\chapter{Alcance}
\label{cap:alcance}

Al ser un proyecto que puede ser complejo y largo, definiremos el alcance y los objetivos que se deberan de cumplir para considerar que las hipotesis que nos
hemos planteado esten satisfechas. 

\section{Objetivos}
\label{sec:objetivos}

Este projecto tiene como objetivo principal investigar el uso de inteligencias artificiales para aplicar ingieneria inversa a programas. 
En otras palabras, nos proponemos utilizar inteligencias artificiales tipo ChatGPT o similares para generar un codigo en C a partir unica
y exclusivamente del ejecutable o de un archivo objeto.

Para poder alcanzar este objetivo se ha decidido dividirlo en subobjetivos de tal manera que podamos ir dando pasos hasta el objetivo principal.
Los subobjetivos definidos son los siguientes:

\begin{itemize}
    \item Pruebas de uso con ChatGPT Plus, utilizando modelos como el GPT-3.5 y GPT-4
    \item Pruebas de uso con un modelo mas pequeño denominados LoRA\footnote{LoRA (\textit{Low-Rank Adaptation}) es una tecnica de \textit{fine-tuning} para modelos
          prentrenados en la cual propone utilizar una matriz de rangos baja de tal manera que reducimos los tiempos y memoria de GPU necesarios para el entrenamiento.} 
          y haciendo \textit{fine-tuning} para nuestra tarea en especifico.
\end{itemize}

El objetivo de realizar pruebas con ChatGPT es obtener una aproximación de los resultados que podamos obtener con nuestro modelo. ChatGPT nos
ofrece un entorno rapido y sencillo para poder hacer pruebas y analizar los resultados que podamos obtener.

Una vez realizadas las pruebas con ChatGPT, independientemente del resultado obtenido, se procedera a desplejar una inteligencia artificial mas pequeña,
denominadas LoRa, en la cual se procedera a hacer diferentes metodos de \textit{fine-tuning}de tal manera que podemos augmentar la calidad de los resultados
obtenidos. La principal razon por la cual utilizamos modelos LoRa se debe a una cuestion tecnica. Estos modelos estan optimizados para obtener resultados
similares a un modelo normal como GPT-3.5 pero utilizando una cantidad menor de recursos hardware.

\section{Posibles riesgos y obstaculos}
\label{sec:riesgos}

Los riesgos que se asumen en este proyecto son altos debido a la incertidumbre que hay sobre este tema. Dado que los antecedentes son practicamente nulos y dada una tecnologia
que esta en auge y aun en desarrollo los posibles problemas o resultados no esperados que nos podamos encontrar durante la investigación y el desarrollo son altos.

Alguno de los posibles obstaculos que nos podemos encontrar son:

\begin{itemize}
    \item \textbf{Resultados no deseados:} podemos encontrarnos que a loa hora de querer generar código C a partir de un ejecutable el modelo no sea capaz de procesar estos datos y por lo tanto nos de una salida que no es la esperada.
    \item \textbf{Insuficiencia de recursos:} los modelos que utilizaremos son modelos muy grandes que necesitat mucha memoria de GPU para poderse ejecutar, aunque sabemos que existen modelos mas pequeños y que necesitan menos recursos, podemos encontrarnos de que no sea suficiente esta reducción.
\end{itemize}