\chapter{Resultados}
\label{cap:resultados}

% Corregido 02/01/2023
% TODO:daniel: Revisar el capítulo de resultados

En este capítulo se presentan los resultados obtenidos en el desarrollo de este trabajo. 
En la sección \ref{sec:evaluacion} se describe la metodología de evaluación utilizada 
para medir el desempeño de los modelos. En la sección \ref{sec:resultados_finales} se
presentan los resultados finales obtenidos con el modelo ChatGPT. Finalmente, en la
sección \ref{sec:analisis_resultados} se realiza un análisis de los resultados obtenidos.

\section{Metodología de evaluación}
\label{sec:evaluacion}

% Corregido 08/01/2024
% TODO:daniel: Revisar la metodología de evaluación

Para poder evaluar la calidad de los resultados se han definido una serie de criterios que
nos permitirán evaluar los resultados obtenidos. Estos criterios son los siguientes:

\begin{enumerate}
    \item \textbf{Criterio 1:} el modelo debe de ser capaz de generar código en C que sea
        compilable y ejecutable.
    \item \textbf{Criterio 2:} el modelo debe de ser capaz de generar código en C que sea
        funcionalmente correcto.
\end{enumerate}

Por lo tanto, si el resultado cumple con el primer criterio, diremos que es un resultado aceptable,
ya que ha generado un código sintácticamente correcto. Si el resultado cumple con el segundo criterio
diremos que es un resultado bueno, ya que ha generado un código sintácticamente correcto y funcionalmente
correcto. Si no cumple con ninguno de los dos criterios, diremos que es un resultado malo.

\section{Resultados finales}
\label{sec:resultados_finales}

% Corregido 09/01/2024
% TODO:daniel: Revisar los resultados finales

Después de haber ejecutado el \textit{fine-tuning} de nuestro modelo, vamos a hacer las mismas pruebas
que hicimos en el capítulo \ref{cap:viabilidad_hipotesis} para comprobar si el modelo es capaz de
generar código en C que sea compilable y ejecutable. Para ello, vamos a utilizar el mismo conjunto
de datos. Diferenciaremos entre los resultados obtenidos en el Clúster del departamento de Arquitectura
de Computadores y los resultados obtenidos en Lightning IA Studio.

\subsection{Cluster AC}
\label{subsec:cluster_ac}



\subsection{Lightning IA Studio}
\label{subsec:lightning_ia_studio}

\section{Análisis de resultados}
\label{sec:analisis_resultados}

Vista general de los resultados obtenidos y análisis de los mismos.