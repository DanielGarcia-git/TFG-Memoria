\chapter{Diseño e implementación del sistema de scripts}
\label{cap:diseñoImplentacion_scripts}

% Corregido 29/12/2023
% TODO:danielg: Revisar la introducción del capítulo

Como hemos visto en el capítulo \ref{cap:objetivos} el objetivo principal de este
proyecto es poder generar código en C a partir de un fichero ejecutable en su forma
de ensamblador. Para poder alcanzar este objetivo nos asistiremos con inteligencia
artificial de tal manera que podamos conseguir resultados óptimos. Para ello deberemos
de generar un \textit{dataset} que contenga los datos necesarios para poder entrenar
nuestro modelo.

Por ello, se ha diseñado un sistema de scripts que nos permitirá automatizar el proceso
de generar nuestro \textit{dataset}. Para el diseño de este sistema se han seguido los
siguientes criterios:

\begin{itemize}
    \item \textbf{Modularidad:} el sistema de scripts debe de ser modular, de tal manera
        que se puedan añadir nuevas funcionalidades de manera sencilla.
    \item \textbf{Mantenibilidad:} el sistema de scripts debe de ser mantenible, de tal
        manera que se puedan corregir errores o añadir nuevas funcionalidades de manera
        sencilla.
    \item \textbf{Portable:} el sistema de scripts debe de ser portable, de tal manera
        que se pueda ejecutar en cualquier sistema operativo.
\end{itemize}

\section{Diseño del sistema de scripts}
\label{sec:diseño_sistema_scripts}

% Corregido 30/12/2023
% TODO:danielg: Revisar la sección de diseño del sistema de scripts

Como bien he mencionado en el apartado anterior \ref{cap:diseñoImplentacion_scripts}, el
sistema de scripts debe de ser modular, mantenible y portable. Podréis ver en la sección
\ref{subsec:diagrama_clases} el diagrama de clases del sistema de scripts. Pero antes de
entrar en detalle de los diferentes módulos implementados, explicaré las funcionalidades
que nos ofrece este sistema de scripts.

\subsection{Casoso de uso del sistema de scripts}
\label{subsec:casos_uso_sistema_scripts}

% Corregido 30/12/2023
% TODO:danielg: Revisar la sección de diseño del sistema de scripts

El sistema de scripts tiene como objetivo principal automatizar la generación de un dataset
que contenga los datos necesarios para poder entrenar nuestro modelo. Para la generación
de este dataset se ha dividido en diferentes pasos:

% TODO:danielg: Revisar si hace falta añadir algun paso mas como por ejemplo en del finetuning

\begin{enumerate}
    \item \textbf{Recolección de datos:} en este paso se recogerán los datos necesarios
        para poder generar el dataset. En mi caso, se trata de recolectar códigos escritos
        en C. Mayoritariamente se han recolectado de la página web \textit{Github}
        \footnote{URL al sitio web \href{https://github.com/}{Github}}, de repositorios 
        públicos que se detallan en el apéndice \ref{apen:repositorios}.
    \item \textbf{Compilación y generación de ensamblador:} en este paso se compilarán los
        códigos obtenidos en el paso anterior y se generará el código en ensamblador. En este
        paso se decidirán cosas como el nivel de optimización de la compilación (con esto podemos
        definir diferentes niveles de complejidad en el código ensamblador). Una vez generado el
        ejecutable se extraerá el código ensamblador utilizando herramientas como \textit{objdump}
        \footnote{\textit{objdump} es un programa de línea de comandos para mostrar 
        diversa información sobre archivos objeto en sistemas operativos tipo Unix. Por ejemplo, 
        puede utilizarse como desensamblador para ver un ejecutable en forma de ensamblador.} o 
        \textit{dumpbin}\footnote{El volcado de archivos binarios COFF de Microsoft o \textit{dumpbin}
        muestra información sobre los archivos binarios del formato de archivo de objeto común.}.
    \item \textbf{Generación de dataset:} en este paso se generará el dataset a partir de los
        códigos ensambladores obtenidos en el paso anterior. Este paso generará un fichero JSON
        \footnote{JSON o JavaScript Object Notation es un formato de texto sencillo para el 
        intercambio de datos.} con los datos necesarios y en el formato correcto para poder 
        entrenar nuestro modelo.
\end{enumerate}

\subsection{Módulos del sistema de scripts}
\label{subsec:modulos_sistema_scripts}

En esta sección se detallarán los diferentes módulos y submódulos definidos en el sistema de scripts y las
funcionalidades o propósitos que tienen cada uno de ellos. Para más información sobre los métodos que contiene
cada módulo, submódulo o clase podéis consultar la documentación generada que se encuentra en conjunto con
el código fuente del proyecto.

Los módulos definidos son los siguientes:

\begin{itemize}
    \item \textbf{Modulo main}
    \begin{itemize}
        \item \textit{main.tasks}
        \item \textit{main.command}
        \item \textit{main.log}
    \end{itemize}
    \item \textbf{Modulo script}
    \begin{itemize}
        \item \textit{script.dataset}
        \item \textit{script.compiler}
        \item \textit{script.repository}
    \end{itemize}
    \item \textbf{Modulo utils}
    \begin{itemize}
        \item \textit{utils.abstract}
        \item \textit{utils.file}
        \item \textit{utils.enum}
    \end{itemize}
\end{itemize}

\subsubsection{Modulo main}
\label{subsubsec:modulo_main}

El módulo \textit{main} es el módulo principal del sistema de scripts. Este módulo se encarga de interpretar
la entrada del usuario (los argumentos) y de llamar a los diferentes submódulos para que realicen las tareas
correspondientes.

En la tabla \ref{tab:main_description} se puede ver una descripción de los diferentes submódulos y clases que
componen el módulo \textit{main}.

\begin{table}[H]
    \centering
    \resizebox{\textwidth}{!}{%
    \begin{tabular}{|l|l|l|l|l|}
    \hline
    \rowcolor[HTML]{8EA9D8} 
    Módulo & Submódulo    & Clase            & Función & Descripción \\ \hline
    main   & main.log     & -                &         &             \\ \hline
    main   & main.log     & LogManager       &         &             \\ \hline
    main   & main.command & -                &         &             \\ \hline
    main   & main.command & Configuration    &         &             \\ \hline
    main   & main.command & CommandProcessor &         &             \\ \hline
    main   & main.tasks   & -                &         &             \\ \hline
    main   & main.tasks   & Version          &         &             \\ \hline
    main   & main.tasks   & Help             &         &             \\ \hline
    main   & main.tasks   & CleanUp          &         &             \\ \hline
    main   & main.tasks   & Compiler         &         &             \\ \hline
    main   & main.tasks   & DataSet          &         &             \\ \hline
    main   & main.tasks   & RepositorySetup  &         &             \\ \hline
    \end{tabular}%
    }
    \caption{Descripción de los submódulos y clases que componen el módulo main}
    \label{tab:main_description}
\end{table}

\subsubsection{Modulo script}
\label{subsubsec:modulo_script}

En la tabla \ref{tab:script_description} se puede ver una descripción de los diferentes submódulos y clases que
componen el módulo \textit{script}.

\begin{table}[H]
    \centering
    \resizebox{\textwidth}{!}{%
    \begin{tabular}{|l|l|l|l|l|}
    \hline
    \rowcolor[HTML]{8EA9D8} 
    Módulo & Submódulo         & Clase           & Función & Descripción \\ \hline
    script & script.datset     & -               &         &             \\ \hline
    script & script.datset     & DataFineTuning  &         &             \\ \hline
    script & script.datset     & DataSet         &         &             \\ \hline
    script & script.compiler   & -               &         &             \\ \hline
    script & script.compiler   & Compiler        &         &             \\ \hline
    script & script.compiler   & CompilerOptions &         &             \\ \hline
    script & script.repository & -               &         &             \\ \hline
    script & script.repository & Repository      &         &             \\ \hline
    script & script.repository & RepositoryList  &         &             \\ \hline
    \end{tabular}%
    }
    \caption{Descripción de los submódulos y clases que componen el módulo script}
    \label{tab:script_description}
\end{table}

\subsubsection{Modulo utils}
\label{subsubsec:modulo_utils}

En la tabla \ref{tab:utils_description} se puede ver una descripción de los diferentes submódulos y clases que
componen el módulo \textit{utils}.

\begin{table}[H]
    \centering
    \resizebox{\textwidth}{!}{%
    \begin{tabular}{|l|l|l|l|l|}
    \hline
    \rowcolor[HTML]{8EA9D8} 
    \multicolumn{1}{|c|}{\cellcolor[HTML]{8EA9D8}Módulo} & \multicolumn{1}{c|}{\cellcolor[HTML]{8EA9D8}Submódulo} & \multicolumn{1}{c|}{\cellcolor[HTML]{8EA9D8}Clase} & \multicolumn{1}{c|}{\cellcolor[HTML]{8EA9D8}Función}                                                       & \multicolumn{1}{c|}{\cellcolor[HTML]{8EA9D8}Descripción}                                                                                                                            \\ \hline
    utils                                                & utils.abstract                                         & -                                                  & Declaración de las clases abstractas                                                                       & \begin{tabular}[c]{@{}l@{}}Este submódulo contendrá todas las \\ clases abstractas utilizadas en el sistema\end{tabular}                                                            \\ \hline
    utils                                                & utils.abstract                                         & Task                                               & Representa una tarea generica                                                                              & \begin{tabular}[c]{@{}l@{}}Esta clase define los métodos que tiene una\\ tarea y obliga implementar aquellos que son\\ específicos de una tarea\end{tabular}                        \\ \hline
    utils                                                & utils.enum                                             & -                                                  & Declaración de los enumeradores                                                                            & \begin{tabular}[c]{@{}l@{}}Este submódulo contendrá todos los\\ enumeradores utilizados en el sistema\end{tabular}                                                                  \\ \hline
    utils                                                & utils.enum                                             & Arguments                                          & Declaración de los argumentos                                                                              & \begin{tabular}[c]{@{}l@{}}En esta clase se definen los argumentos \\ disponibles en el sistema con una breve descripción\end{tabular}                                              \\ \hline
    utils                                                & utils.enum                                             & CompilerOptions                                    & \begin{tabular}[c]{@{}l@{}}Declaración de las opciones del\\ compilador\end{tabular}                       & \begin{tabular}[c]{@{}l@{}}En esta clase se definen las opciones que el\\ compilador tiene disponible\end{tabular}                                                                  \\ \hline
    utils                                                & utils.enum                                             & Compilers                                          & \begin{tabular}[c]{@{}l@{}}Declaración de los compiladores\\ disponibles\end{tabular}                      & \begin{tabular}[c]{@{}l@{}}En esta clase se definen los compiladores y su\\ comando que se tienen disponibles en el sistema\end{tabular}                                            \\ \hline
    utils                                                & utils.enum                                             & OperatingSystem                                    & \begin{tabular}[c]{@{}l@{}}Declaración de los sistemas\\ operativos soportados\end{tabular}                & \begin{tabular}[c]{@{}l@{}}En esta clase se definen los sistemas operativos \\ soportados por el sistema\end{tabular}                                                               \\ \hline
    utils                                                & utils.enum                                             & Paths                                              & Declaración de los paths utilizados                                                                        & \begin{tabular}[c]{@{}l@{}}En esta clase se declaran los paths que se utilizan\\ como por ejemplo donde están ubicados los logs\end{tabular}                                        \\ \hline
    utils                                                & utils.enum                                             & RepositoryType                                     & \begin{tabular}[c]{@{}l@{}}Declaración de los tipos de\\ repositorios\end{tabular}                         & \begin{tabular}[c]{@{}l@{}}En esta clase se declaran los tipos de repositorios \\ que el sistema tiene, por ejemplo, repositorio de\\ código, de un sistema de IA, etc\end{tabular} \\ \hline
    utils                                                & utils.file                                             & -                                                  & \begin{tabular}[c]{@{}l@{}}Declaración de las clases utilizadas\\ para representar un archivo\end{tabular} & \begin{tabular}[c]{@{}l@{}}Este submódulo contendrá todas las clases\\ que representen un archivo por ejemplo\\ de código\end{tabular}                                              \\ \hline
    utils                                                & utils.file                                             & File                                               & Declaración de un fichero generico                                                                         & \begin{tabular}[c]{@{}l@{}}Esta clase representa un fichero genérico que\\ contiene cualquier tipo de datos\end{tabular}                                                            \\ \hline
    utils                                                & utils.file                                             & CodeFile                                           & \begin{tabular}[c]{@{}l@{}}Declaración de un fichero que\\ contiene código\end{tabular}                    & \begin{tabular}[c]{@{}l@{}}Esta clase representa un fichero de código escrito\\ en cualquier lenguaje de programación\end{tabular}                                                  \\ \hline
    utils                                                & utils.file                                             & ObjdumpFile                                        & \begin{tabular}[c]{@{}l@{}}Declaración de un fichero que\\ contienen código en ensamblador\end{tabular}    & \begin{tabular}[c]{@{}l@{}}Esta clase representa un fichero de la salida estándar\\ al ejecutar un comando del estilo objdump\end{tabular}                                          \\ \hline
    \end{tabular}%
    }
    \caption{Descripción de los submódulos y clases que componen el módulo utils}
    \label{tab:utils_description}
\end{table}

\newpage
\paperwidth=\pdfpageheight
\paperheight=\pdfpagewidth
\pdfpageheight=\paperheight
\pdfpagewidth=\paperwidth
\headwidth=\textheight

\subsection{Diagrama de clases}
\label{subsec:diagrama_clases}

% TODO:danielg: Comprobar que el UML corresponde al diseño final

\begin{figure}[H]
    \begin{center}
        \includegraphics[scale=0.2]{figuras/Capitulo_XX/UML_Script.png}
    \end{center}
    \caption[Diagrama de clases del sistema de scripts]{Diagrama de clases del sistema de scripts (Elaboración propia)}
    \label{fig:UML_Script}
\end{figure}\

\newpage
\paperwidth=\pdfpageheight
\paperheight=\pdfpagewidth
\pdfpageheight=\paperheight
\pdfpagewidth=\paperwidth
\headwidth=\textwidth

\section{Implementación del sistema de scripts}
\label{sec:implementacion_sistema_scripts}

