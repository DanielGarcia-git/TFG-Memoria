\chapter{Fases de la Ingenieria inversa asistida por IA}
\label{cap:hoja_de_ruta}

% Corregido 07/01/2024
% Revisado 14/01/2024

En este capítulo se presenta la hoja de ruta que se ha seguido para el desarrollo de
este proyecto. El proyecto se ha dividido en diferentes fases:

\begin{enumerate}
    \item \textbf{Fase 1:obtención de resultados preliminares para ver la viabilidad 
        de la hipótesis planteada.} 

        En la fase 1, y la cual se detalla en el capítulo \ref{cap:viabilidad_hipotesis},
        se ha realizado un estudio preliminar utilizando modelos de lenguaje
        disponibles en internet y de los cuales puedes interactuar con ellos a través de
        una página web o API. El modelo utilizado será GPT-4 a través de su interfaz web
        ChatGPT. El objetivo de esta fase es ver si los resultados obtenidos son prometedores
        y si la hipótesis planteada es viable. De tal manera que podamos extrapolar los resultados
        a un modelo que ejecutare de manera local y que nos permitirá obtener mejores resultados
        aplicándole un \textit{fine-tuning}.

    \item \textbf{Fase 2: investigación y análisis de los modelos de lenguaje disponibles,
        así como definición de la estrategia de entrenamiento que se va a seguir.} 
        
        En la fase 2, y la cual se detalla en el capítulo \ref{cap:estrategia_entrenamiento},
        se ha realizado un estudio e investigación de los modelos de lenguaje disponibles
        gratuitamente en internet. Se ha analizado las características de cada uno de ellos
        teniendo muy en cuenta su tamaño y el coste computacional que supone entrenarlos, ya que
        en nuestro proyecto es una limitación que tenemos que tener muy en cuenta. También se ha
        definido la estrategia de entrenamiento que se va a seguir.

    \item \textbf{Fase 3: diseño e implementación de los scripts necesarios para la
        extracción de los datos de entrenamiento y construcción de los conjuntos de entrenamiento.} 
        
        En la fase 3, y la cual se detalla en el capítulo \ref{cap:diseñoImplentacion_scripts},
        se ha realizado el diseño e implementación de los scripts necesarios para la
        creación de un conjunto de entrenamiento. Este conjunto de entrenamiento se
        compone de un conjunto de datos de entrada, el código en ensamblador, y un conjunto
        de datos de salida, el código en C.

    \item \textbf{Fase 4: configuración de los entornos de entrenamiento y ejecución de
        los mismos.} 
        
        En la fase 4, y la cual se detalla en el capítulo \ref{cap:configuracion_ejecucion},
        se ha realizado la configuración de los entornos de entrenamiento y se ha ejecutado
        el entrenamiento de los modelos de lenguaje con el conjunto de entrenamiento generado.

    \item \textbf{Fase 5: análisis de los resultados obtenidos y conclusiones.} 

        En la fase 5, y la cual se detalla en el capítulo \ref{cap:resultados},
        se ha realizado un análisis de los resultados obtenidos y se han analizado
        los resultados obtenidos.
\end{enumerate}