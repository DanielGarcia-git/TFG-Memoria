\chapter{Estado del arte de la ingeniería inversa}
\label{cap:estadoDelArte}

\begin{flushright}
    \begin{minipage}[]{10cm}
        \emph{Sometimes, the best way to advance is in reverse}\\
    \end{minipage}\\

    Eldad Eilam, \textit{Reversing : secrets of reverse engineering}\\
\end{flushright}

\vspace{1cm}

% TODO:daniel: Corregir y revisar El estado del arte de la ingeniería inversa

Antes de entrar en profundidad en el proyecto que en esta memoria se detalla me
gustaría introducir formalmente el concepto de ''ingeniería inversa'' o conocida
en el mundo de la informática como \textit{reverse engineering}. La ingeniería
inversa es el proceso de extraer conocimiento o el diseño por cualquier cosa que 
el humano haya hecho. Por lo tanto, no es un concepto que solo acotemos dentro de
la ingeniería informática, sino que se puede aplicar a cualquier proceso ingenieril.
De hecho, este proceso es bastante similar al método científico, con la única diferencia
que la ingeniería inversa solo se aplica sobre cosas que ha hecho un humano y en el
método científico lo aplicamos en fenómenos naturales.

La ingeniería inversa es usada normalmente para obtener conocimiento desconocido o la
filosofía del diseño cuando esta información no está disponible, ya sea porque su
propietario ha decidido no compartirla o porque esta información está perdida o 
destruida. \cite{alma991003132729706711}

La ingeniería inversa aplicada al \textit{software} es el mismo concepto, pero aplicada
a programa en su formato binario, con el objetivo de obtener el código fuente en un
lenguaje de programación concreto y así poder obtener información como el diseño del
programa. De hecho, la ingeniería inversa aplicada al \textit{software} requiere
diferentes artes: descifrado de códigos, resolución de puzzles, programación y análisis
lógico.

Las aplicaciones de la ingeniería inversa son muy variadas, pero podemos destacar dos
categorías: seguridad y desarrollo de \textit{software}.

Sus aplicaciones en seguridad son muy variadas, pero normalmente se le relaciona con
\textit{malware}, algoritmos criptográficos y auditorias sobre binarios.

Dentro del mundo del \textit{software} malicioso vemos que la ingeniería inversa se
utiliza en dos aspectos diferentes. Desde el punto de perspectiva de los que desarrollan
\textit{software} malicioso utilizan la ingeniería inversa para poder encontrar
vulnerabilidades en los programas que quieren infectar. En cambio, desde el punto de vista
de los desarrolladores de antivirus, utilizan la ingeniería inversa para poder diseccionar
y analizar cada programa malicioso.

También se puede aplicar ingeniería inversa sobre algoritmos criptográficos, de tal manera
que podamos averiguar que tan seguro es ese mensaje encriptado, o incluso en caso de utilizar
algoritmos basados en llaves, en los cuales la especificación del algoritmo de encriptación
es conocido, pero que cada implementación específica puede variar, encontrar vulnerabilidades
en estos algoritmos.

Y por último, también la ingeniería inversa se aplica para realizar auditorias sobre binarios,
de tal manera que podamos detectar si un programa es seguro o no, encontrar sus vulnerabilidades
para poder corregirlas. Por lo tanto, cuanto mejores sean las herramientas de ingeniería inversa
que se apliquen, podremos encontrar con mucha más eficacia problemas de seguridad en \textit{software}
propietario.

Como he mencionado con anterioridad, la ingeniería inversa también se aplica dentro del desarrollo
de \textit{software}, estas aplicaciones las podemos encontrar en diferentes etapas por ejemplo
cuando disponemos de un \textit{software} propietario y a documentación es escasa, el uso de
herramientas de ingeniería inversa nos podrían ayudar a conseguir más interoperabilidad con
el \textit{software} propietario.

Otras de las aplicaciones, y que he mencionado anteriormente, es el comprobar la robustez y calidad
de un programa informático, de tal manera que se pueda corregir posibles problemas de seguridad o
problemas funcionales.

Una vez introducido el concepto de ingeniería inversa de manera general, creo que es interesante
contestar a la siguiente pregunta: \textbf{¿Es la ingeniería inversa legal?}

Como hemos podido ver en los párrafos anteriores, muchas veces aplicamos ingeniería inversa sobre
un \textit{software} propietario, algunos con fines de conocer vulnerabilidades y corregirlas
y otros con fines de encontrar estas vulnerabilidades para tener un punto de ataque. En
otras palabras, la legalidad de la aplicación de ingeniería inversa dependerá mucho
de para que lo estamos aplicando, la finalidad, y sobre que lo estamos aplicando.

En conclusión, la ingeniería inversa es un proceso complejo y que requiere de muchas habilidades 
y que sus aplicaciones son muy variadas, desde auditorias sobre binarios hasta mejorar
la interoperabilidad de dos programas.

\section{Terminología y definiciones}
\label{sec:terminalogia}

% TODO:daniel: Corregir y revisar la terminalogia

En este apartado pasaremos a detallar ciertos términos reiterativos que veremos a lo
largo del proyecto y que son de especial importancia. Las definiciones son las siguientes:

\begin{itemize}
    \item \textbf{Large Lenguage Model:} un LLM o \textit{Large Lenguage Model} son 
        modelos \textit{Pre-trained Lenguage Models}\footnote{El concepto de pre
        entrenamiento en un modelo de lenguaje está relacionado con el aprendizaje
        por transferencia. La idea del aprendizaje por transferencia es reutilizar
        los conocimientos aprendidos en una o varias tareas y aplicarlos a tareas
        nuevas.} (PLM) a los cuales se han aumentado o el tamaño del modelo en sí
        o los datos. Con este aumento se dieron cuenta de que había una notoria mejora
        en términos de \textit{performance} y de la capacidad de los modelos en hacer
        ciertas tareas. \cite{ZhaoWayneXin2023ASoL}
    \item \textbf{Fine-tuning:} es una técnica de entrenamiento que consiste en
        reutilizar un modelo predefinido y preentrenado, de tal manera que ajustamos
        ciertas capas de la red neuronal para obtener mejores resultados para nuestra
        tarea en específico.
\end{itemize}