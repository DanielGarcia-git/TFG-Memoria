\chapter{Informe de sostenibilidad}
\label{cap:sostenibilidad}

La sostenibilidad de un proyecto dentro de un contexto de ingineria informatica es de vital importancia, debido a que el objetivo de un ingienero informatico 
es encontrar el equilibrio entre algo que es util para la sociedad, es rentable y a la vez es sostenible, en terminos ambientales, en el tiempo.

En mi caso, siempre he sido muy consciente de encontrar este equilibrio entre estos tres aspectos y sobretodo en encontrar un metodo sostenible medioambientamente. Además, en
un proyecto de estas caracteristicas tambien me ha supuesto un reto a nivel de sostenibilidad social, ya que en nuestras hipotesis planteamos diferentes
ideas que entrarian en conflicto con la actual industri del software.

Para ello, en este capitulo detallare la sostenibilidad de este proyecto en los tres ambitos mas relevantes: el economico, el ambiental y el social.

\section{Sostenibilidad económica}
\label{sec:sostenibilidad_economica}

Como hemos podido ver el en capitulo \ref{cap:presupuesto} se han detallado los costes economicos de los recursos utilizados, a la hora de realizar este presupuesto se ha tenido
en cuenta que es un proyecto que actualmente despierta mucho interes dentro de la industria debido a la gran alza de las inteligencias artificiales.

Así mismo, este proyecto intenta ver el uso de las inteligencias artificales dentro de una tarea especifica, es decir, estamos atacando sobre un mercado reducido y muy especifico
pero que a la vez trae consigo mucha relevancia e importancia, dado que sus consecuencias hacen que nos planteemos la seguridad de nuestras aplicaciones a todos los niveles.

\section{Sostenibilidad ambiental}
\label{sec:sostenibilidad_ambiental}

Cabe destacar la importancia de la sostenibilidad ambiental en este proyecto, debido a que este, aunque no lo parezca al principio, tiene un gran impacto medioambiental, debido
a que utilizamos inteligencias artificiales que precisan de mucha potencia de calculo, lo cual implica un gasto de energia importante, tanto para alimentar las propias maquinas
que computan estos algoritmos, como los sistemas auxiliares que ayudan a que estos sistemas funcionen.

Para que nos hagamos una idea, se estima que el entrenamiento de un PLN basado en \textit{deep learning} produce unas 284 toneladas de CO2, segun la universidad de Massachusetts.
\cite{Artículo_doctrinal}. Aunque la cifra es una estimación somos conscientes que los grandes centros de procesamiento donde se entrenan estos modelos consumen muchos recursos y por lo tanto, a 
la hora de utilizar modelos como los de ChatGPT hemos de tener en cuenta el impacto ambiental que provocamos con su uso.

En el proyecto se ha tenido en cuenta estos factores y se decidió utilizar modelos mas pequeños y que consumen muchos menos recursos energeticos, ya que estos pueden ser entrenados
en tarjetas graficas comerciales, que aunque consumen energia, es mucho menos que un gran centro de computación.

\section{Sostenibilidad social}
\label{sec:sostenibilidad_social}

Por lo que respecta la sostenibilidad social en el ambito personal este proyecto supone un reto, ya que salgo de lo que normalmente hago y me adentro en un ambito que 
para mi es desconocido, pero que a la vez me supone una gran curiosidad.

Así mismo, este proyecto supone ciertos cambios en el paradigma social, ya que hasta ahora (y si los resultdados son los esperados) no hay ninguna herramienta que sea
capaz de aplicar ingineria inversa sobre programas y generar codigo que sea compilable. Lo que teniamos hasta ahora son herramientas que dan una especie de pseudocodigo que 
requiere de mucha intervención humana tan solo para hacerlo entendible.

Para poder evaluar si hay una necesidad real de una herramienta como la que se describe en este proyecto, me fundamento en la expectación que a generado entre diferentes docentes, dentro y
fuera de este centro, del que han mostrado interes en saber los resultados que se extraen de este proyecto.
